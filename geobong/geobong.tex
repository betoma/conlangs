\documentclass[a4paper,11pt,oneside,openany]{memoir}

\usepackage{fontspec}
\usepackage{amsmath} % a pretty standard package to enlarge your inventory of symbols
\usepackage{amssymb} % another common package for symbols
\usepackage[hidelinks]{hyperref} % enables hyperlinks in your document (no worries -- they show up only on the screen. When you print a hard copy, the colored boxes aren't there)
\usepackage{url} % helps typeset URLs properly, typically with the command \url
\usepackage[margin=.75in]{geometry} % page layout
\usepackage{tikz}
\usepackage{tikz-qtree}
\usepackage{wrapfig}
\usepackage{subcaption}
\usepackage{booktabs} % creates beautiful and professional tables
\usepackage{multicol}
\usepackage{multirow}
\usepackage{textcomp}
\usepackage{expex}
\usepackage{enumitem}
\usepackage{threeparttable}
\usepackage[calc,english]{datetime2}
\usepackage{suffix}
\usepackage{afterpage}
\usepackage{bookmark}
\usepackage{blindtext}
\usepackage{phonrule}
\usepackage[glosses,indexonlyfirst,nonumberlist,toc,nomain,mcolblock,,nogroupskip]{leipzig}
\usepackage{etoolbox}

\setmainfont[StylisticSet=07]{Libertinus Serif}
%\setmainfont{Charis SIL}[CharacterVariant=43:1]

%---generic symbols---%
\newcommand{\dao}{$\to$}
\newcommand{\nm}{\symbol{"2205}}
\newcommand{\ra}{\textgreater}
\newcommand{\ipkt}{·}
\newcommand{\til}{\textasciitilde}
\newcommand{\langbr}{⟨}
\newcommand{\rangbr}{⟩}

\newcommand{\ortho}[1]{$\langle$#1$\rangle$}
\newcommand{\bripa}[1]{[#1]}
\newcommand{\phipa}[1]{/#1/}
\newcommand{\eng}[1]{`#1'}

%---IPA---%
%consonants
\newcommand{\bilaf}{ɸ}
\newcommand{\bilav}{β}
\newcommand{\tht}{θ}
\newcommand{\labrox}{ʋ}
\newcommand{\latfric}{ɬ}
\newcommand{\latfrivoic}{ɮ}
\newcommand{\darkl}{ɫ}
\newcommand{\alvr}{ɹ}
\newcommand{\alvrap}{ɾ}
\newcommand{\alvlap}{ɺ}
\newcommand{\esh}{ʃ}
\newcommand{\ezh}{ʒ}
\newcommand{\alvpalesh}{ɕ}
\newcommand{\alvpalezh}{ʑ}
\newcommand{\paljstop}{ɟ}
\newcommand{\paljfric}{ʝ}
\newcommand{\egna}{ɲ}
\newcommand{\retesh}{ʂ}
\newcommand{\retezh}{ʐ}
\newcommand{\retna}{ɳ}
\newcommand{\egh}{ɣ}
\newcommand{\engma}{ŋ}
\newcommand{\vell}{ʟ}
\newcommand{\velr}{ʁ}
\newcommand{\velprox}{ɰ}
\newcommand{\uvux}{χ}
\newcommand{\pharox}{ʕ}
\newcommand{\glotstop}{ʔ}
%vowels
\newcommand{\frno}{ø}
\newcommand{\bari}{ɨ}
\newcommand{\unru}{ɯ}
\newcommand{\unro}{ɤ}
\newcommand{\eps}{ɛ}
\newcommand{\oeps}{œ}
\newcommand{\sche}{ɘ}
\newcommand{\schwa}{ə}
\newcommand{\centruh}{ɜ}
\newcommand{\opno}{ɔ}
\newcommand{\aesh}{æ}
\newcommand{\oesh}{ɶ}
\newcommand{\centra}{ɐ}
\newcommand{\ahoh}{ɒ}
%diacritics and modifiers
\newcommand{\asp}{ʰ}
\newcommand{\lab}{ʷ}
\newcommand{\pal}{ʲ}
\newcommand{\jekt}{ʼ}
\newcommand{\nav}{̃}
\newcommand{\rhot}{˞}
\newcommand{\sylb}{̩}
\newcommand{\vless}{̥}
\newcommand{\upvless}{̊}
\newcommand{\bck}{̠}
\newcommand{\dwnwrd}{̞}
\newcommand{\upwrd}{̝}
\newcommand{\lamino}{̻}
\newcommand{\apico}{̺}
\newcommand{\lowap}{̞}
\newcommand{\prstr}{ˈ}
\newcommand{\scstr}{ˌ}
\newcommand{\tiebar}{͡}
\newcommand{\lgth}{ː}
\newcommand{\linglab}{̼}
%tone letters
\newcommand{\toneH}{˥}
\newcommand{\toneM}{˧}
\newcommand{\toneL}{˩}
\newcommand{\toneMH}{˦}
\newcommand{\toneML}{˨}

\definelingstyle{default}{%
    glstyle=nlevel,%
    numoffset=3em,%
    textoffset=1.5em,%
    exskip=.75ex,%
    belowglpreambleskip=.5ex,%
    aboveglftskip=.5ex,%
    everyglft=\it,%
    everygl={\parskip=0pt},%
    }%

\lingset{lingstyle=default}

\DTMnewdatestyle{eurodate}{%
    \renewcommand{\DTMdisplaydate}[4]{%
        \number##3.\nobreakspace%           day
        \DTMmonthname{##2}\nobreakspace%    month
        \number##1%                         year
    }%
    \renewcommand{\DTMDisplaydate}{\DTMdisplaydate}%
}

\DTMsetdatestyle{eurodate}

%-----DICT COMMANDS------
%------------------------

\makeatletter
\@beginparpenalty=10000
\makeatother

\newcounter{dictwordcount}
\newcounter{definition}

\newenvironment{entrylist}{
    \begin{description}[leftmargin=*]
    }{
    \end{description}
    }

% \makeatletter
% \newenvironment{simplentry}[1]%
%     {%
%     \item[#1]\hfill
%     \protected@edef\@currentlabelname{#1}%
%     \setcounter{definition}{0}%
%     \begin{description}[align=right,labelwidth=*,font=\normalfont]
%     }{%
%     \end{description}
%     }%
% \makeatother

\makeatletter
\newenvironment{dictentry}[3][]%
    {%
    \item[#2]\ifblank{#1}{\hfill}{\:\bripa{#1}\hfill}\ifblank{#3}{}{\\
    {\footnotesize #3}
    }
    \protected@edef\@currentlabelname{#2}%
    \setcounter{definition}{0}%
    \refstepcounter{dictwordcount}%
    \begin{description}[align=right,labelwidth=*,font=\normalfont]
    }{%
    \end{description}
    }%
\makeatother

\newenvironment{entrysublist}{
    \vspace{2ex}
    \item[]\textit{Compounds \& Phrasal Forms}
    \begin{entrylist}
        \small
    }{
    \end{entrylist}
    }

\newbox\BETposbox
\newdimen\BETcacheleft
\newdimen\BETcachewd

\makeatletter
\newcommand{\dictdef}[2][]{%
    \refstepcounter{definition}%
    \ifblank{#1}{%
    \item[\thedefinition.] #2
    }{%
    \setbox\BETposbox\hbox{\textit{#1}\,\:}%
    \itemindent-\wd\BETposbox
    \BETcacheleft\parshapeindent\@ne
    \BETcachewd\parshapelength\@ne
    \advance\BETcacheleft\wd\BETposbox
    \advance\BETcachewd-\wd\BETposbox
    \parshape\@ne\BETcacheleft\BETcachewd
    \advance\BETcacheleft-\wd\BETposbox
    \advance\BETcachewd\wd\BETposbox
    \item[\thedefinition.]
    \usebox\BETposbox#2\par
    \parshape\@ne\BETcacheleft\BETcachewd
    \itemindent\z@
    }%
}%
\makeatother

% \newcommand{\dictdef}[2][]{\refstepcounter{definition}%
%     \item[\thedefinition.] \ifthenelse{\isempty{#1}}{
%         #2
%     }{
%         \textit{#1}\: #2
%     }%
%     }%
\makeatletter
\WithSuffix\newcommand\dictdef*[2][]{%
    \refstepcounter{definition}%
    \ifblank{#1}{%
    \item[] #2
    }{%
    \setbox\BETposbox\hbox{\textit{#1}\,\:}%
    \itemindent-\wd\BETposbox
    \BETcacheleft\parshapeindent\@ne
    \BETcachewd\parshapelength\@ne
    \advance\BETcacheleft\wd\BETposbox
    \advance\BETcachewd-\wd\BETposbox
    \parshape\@ne\BETcacheleft\BETcachewd
    \advance\BETcacheleft-\wd\BETposbox
    \advance\BETcachewd\wd\BETposbox
    \item[]
    \usebox\BETposbox#2\par
    \parshape\@ne\BETcacheleft\BETcachewd
    \itemindent\z@
    }%
}%
\makeatother

%----------------------------
%----------------------------

\newcommand{\lang}{Geoboŋ}

\newcommand{\gob}[1]{``#1''}

%---LANG'S ORTHO---%
\newcommand{\vc}{č}
\newcommand{\vz}{ž}
\newcommand{\vs}{š}
\newcommand{\vC}{Č}
\newcommand{\vZ}{Ž}
\newcommand{\vS}{Š}
\newcommand{\Engma}{Ŋ}
\newcommand{\okina}{ʻ}


%title page
\newlength{\drop}% for my convenience
\newcommand*{\titleP}{\begingroup%
\drop = 0.12\textheight
\vspace*{\drop}
\hspace*{0.3\textwidth}
{\HUGE\sc \lang}\\[\baselineskip]
\hspace*{0.33\textwidth}
{\huge A Grammar}\par
\vspace*{3\drop}
{\large By {\sc Bethany E. Toma}}
\vfill
{\today}
\vspace*{0.5\drop}
\endgroup}

%----------------------------
%---------Glossaries---------
%----------------------------

\makenoidxglossaries

\newleipzig{test}{tst}{It's a test!}

\glssetwidest{CABBA}

\makeatletter
\newglossarystyle{fixed-mcols}{%
    \setglossarystyle{alttree}%
    \renewenvironment{theglossary}%
    {%
        \begin{multicols}{3}%
            \def\@gls@prevlevel{-1}%
%           \mbox{}\par
        }%
        {\par\end{multicols}}%
}
\makeatother

\maxsecnumdepth{subsection}

\renewcommand{\arraystretch}{1.2}
\setlength{\columnsep}{2em}
\nonzeroparskip

\begin{document}

\begin{titlingpage}
\titleP
%\clearpage
\end{titlingpage}
\frontmatter

\pagestyle{headings}

\chapter{Background \& Motivation}

Geobo{\engma} is a conlang spoken by a race of goblins in my as-of-yet-unnamed conworld. As goblins are sort of an underclass, their language will always be in a semi-precarious diglossic situation, and in the future I hope to add some fun sociolinguistics and diachronics based on interactions with the other languages from this conworld.

This language is intended to have some deliberately weird elements yet still retain a veneer of quasi-naturalism---while naturalism is definitely a concern, it can be discarded for the sake of trying out something fun.

\clearpage
\tableofcontents

\setglossarystyle{fixed-mcols}

\printnoidxglossary[type=\leipzigtype,title={Glossing Abbreviations}]

\chapter{World \& Culture}

To-do. \ToFirst \ToGob

\mainmatter

\part{Grammar}

\chapter{Phonology}

\section{Phoneme Inventory}\label{sec:phon_inv}

\subsection{Consonants}\label{ssec:consonant_inv}

\begin{table}[ht]
    \centerfloat
    \begin{tabular}{@{}rccccccccc@{}}
    \toprule
     &
      \multirow{2}{*}{Bilabial} &
      \multicolumn{2}{c}{Denti-Alveolar} &
      \multicolumn{2}{c}{Apicoalveolar} &
      \multirow{2}{*}{Palatoalveolar} &
      \multirow{2}{*}{Alveolopalatal} &
      \multirow{2}{*}{Velar} &
      \multirow{2}{*}{Glottal} \\
              &   & Non-Sib.      & Sib.    & Non-Sib.     & Sib.   &      &           &        &  \\ \midrule
    Nasal     & m & n\lamino      &         & n\apico      &        &      &           & \engma &  \\
    Plosive &
      p b &
      t\lamino\;d\lamino &
      t\tiebar s\lamino\;d\tiebar z\lamino &
      t\apico\;d\apico &
      t\tiebar s\apico\;d\tiebar z\apico &
      t\tiebar\esh\;d\tiebar\ezh &
      t\tiebar\alvpalesh\;d\tiebar\alvpalezh &
      k g &
      \glotstop \\
    Fricative &   & \latfric\lamino & s\lamino & \latfric\apico & s\apico & \esh       & \alvpalesh      &  &  \\
    Approx.   &   & l\lamino        &          & l\apico        &          & \alvr\bck & \alvr\bck\pal   &  &  \\ \bottomrule
    \end{tabular}
    \caption{Consonant Inventory}
    \label{tab:consonants}
\end{table}

\lang{}'s consonant inventory is somewhat large but fairly regular. By far the bulk of the inventory consists of coronals, comprising twenty-four of the thirty-one consonants. \lang{} also distinguishes between a fairly large number of features within the coronals. 

\begin{figure}[bt]
    \centering
    \footnotesize
    \begin{forest}
        [{[coronal]}
            [{[laminal]}, edge label={node[midway,above left,font=\scriptsize]{+}}
                [{[nasal]}, edge label={node[midway,above left,font=\scriptsize]{+}}
                    [{n\lamino}, edge label={node[midway,above left,font=\scriptsize]{+}}]
                    [{[approximant]}, edge label={node[midway,above right,font=\scriptsize]{--}}
                        [{[anterior]}, edge label={node[midway,above left,font=\scriptsize]{+}}
                            [{l\lamino}, edge label={node[midway,above left,font=\scriptsize]{+}}]
                            [{\alvr\bck\lamino\pal}, edge label={node[midway,above right,font=\scriptsize]{--}}]
                        ]
                        [{[sibilant]}, edge label={node[midway,above right,font=\scriptsize]{--}}
                            [{[closed]}, edge label={node[midway,above left,font=\scriptsize]{+}}
                                [{[continuant]}, edge label={node[midway,above left,font=\scriptsize]{+}}
                                    [{s\lamino}, edge label={node[midway,above left,font=\scriptsize]{+}}]
                                    [{[voiced]}, edge label={node[midway,above right,font=\scriptsize]{--}}
                                        [{t\tiebar s\lamino}, edge label={node[midway,above left,font=\scriptsize]{+}}]
                                        [{d\tiebar z\lamino}, edge label={node[midway,above right,font=\scriptsize]{--}}]
                                    ]
                                ]
                                [{[continuant]}, edge label={node[midway,above right,font=\scriptsize]{--}}
                                    [{\alvpalesh}, edge label={node[midway,above left,font=\scriptsize]{+}}]
                                    [{[voiced]}, edge label={node[midway,above right,font=\scriptsize]{--}}
                                        [{t\tiebar\alvpalesh}, edge label={node[midway,above left,font=\scriptsize]{+}}]
                                        [{d\tiebar\alvpalezh}, edge label={node[midway,above right,font=\scriptsize]{--}}]
                                    ]
                                ]
                            ]
                            [{[continuant]}, edge label={node[midway,above right,font=\scriptsize]{--}}
                                [{\latfric\lamino}, edge label={node[midway,above left,font=\scriptsize]{+}}]
                                [{[voiced]}, edge label={node[midway,above right,font=\scriptsize]{--}}
                                    [{t\lamino}, edge label={node[midway,above left,font=\scriptsize]{+}}]
                                    [{d\lamino}, edge label={node[midway,above right,font=\scriptsize]{--}}]
                                ]
                            ]
                        ]
                    ]
                ]
                [{[nasal]}, edge label={node[midway,above right,font=\scriptsize]{--}}
                    [{n\apico}, edge label={node[midway,above left,font=\scriptsize]{+}}]
                    [{[approximant]}, edge label={node[midway,above right,font=\scriptsize]{--}}
                        [{[anterior]}, edge label={node[midway,above left,font=\scriptsize]{+}}
                            [{l\apico}, edge label={node[midway,above left,font=\scriptsize]{+}}]
                            [{\alvr\bck\apico}, edge label={node[midway,above right,font=\scriptsize]{--}}]
                        ]
                        [{[sibilant]}, edge label={node[midway,above right,font=\scriptsize]{--}}
                            [{[closed]}, edge label={node[midway,above left,font=\scriptsize]{+}}
                                [{[continuant]}, edge label={node[midway,above left,font=\scriptsize]{+}}
                                    [{s\apico}, edge label={node[midway,above left,font=\scriptsize]{+}}]
                                    [{[voiced]}, edge label={node[midway,above right,font=\scriptsize]{--}}
                                        [{t\tiebar s\apico}, edge label={node[midway,above left,font=\scriptsize]{+}}]
                                        [{d\tiebar z\apico}, edge label={node[midway,above right,font=\scriptsize]{--}}]
                                    ]
                                ]
                                [{[continuant]}, edge label={node[midway,above right,font=\scriptsize]{--}}
                                    [{\esh}, edge label={node[midway,above left,font=\scriptsize]{+}}]
                                    [{[voiced]}, edge label={node[midway,above right,font=\scriptsize]{--}}
                                        [{t\tiebar\esh}, edge label={node[midway,above left,font=\scriptsize]{+}}]
                                        [{d\tiebar\ezh}, edge label={node[midway,above right,font=\scriptsize]{--}}]
                                    ]
                                ]
                            ]
                            [{[continuant]}, edge label={node[midway,above right,font=\scriptsize]{--}}
                                [{\latfric\apico}, edge label={node[midway,above left,font=\scriptsize]{+}}]
                                [{[voiced]}, edge label={node[midway,above right,font=\scriptsize]{--}}
                                    [{t\apico}, edge label={node[midway,above left,font=\scriptsize]{+}}]
                                    [{d\apico}, edge label={node[midway,above right,font=\scriptsize]{--}}]
                                ]
                            ]
                        ]
                    ]
                ]
            ]
            [{[labial]}, , edge label={node[midway,above right,font=\scriptsize]{--}}
                [{[nasal]}, edge label={node[midway,above left,font=\scriptsize]{+}}
                    [{m}, edge label={node[midway,above left,font=\scriptsize]{+}}]
                    [{[voice]}, edge label={node[midway,above right,font=\scriptsize]{--}}
                        [{b}, edge label={node[midway,above left,font=\scriptsize]{+}}]
                        [{p}, , edge label={node[midway,above right,font=\scriptsize]{--}}]
                    ]
                ]
                [{[dorsal]}, edge label={node[midway,above right,font=\scriptsize]{--}}
                    [{[nasal]}, edge label={node[midway,above left,font=\scriptsize]{+}}
                        [{\engma}, edge label={node[midway,above left,font=\scriptsize]{+}}]
                        [{[voice]}, edge label={node[midway,above right,font=\scriptsize]{--}}
                            [{g}, edge label={node[midway,above left,font=\scriptsize]{+}}]
                            [{k}, edge label={node[midway,above right,font=\scriptsize]{--}}]
                        ]
                    ]
                    [{\glotstop}, edge label={node[midway,above right,font=\scriptsize]{--}}
                    ]
                ]
            ]
        ]
    \end{forest}
    \caption{Contrastive Hierarchy of \lang{} Consonants}
    \label{fig:cons_hierarchy}
\end{figure}


Laminarity appears to be the most fundamental of these distinctions, as all coronals can be classified into laminal and alveolar classes, and each has a counterpart in the other class. Among and overlapping with these classes are two major classes based on place of articulation---the frontmost coronals at the alveolar ridge and the post-alveolar coronals. The laminal alveolar coronals are realized as laminal denti-alveolar, while naturally the apical alveolar coronals are apico-alveolar. Among the postalveolar coronals, the alveolopalatals represent the laminals, whereas the palato-alveolars are grouped with the apical class (though in actuality it is not articulated apically).

A voicing distinction is made across all stops, with the exception of the glottal stop. However, this distinction is made only across stops---all sonorants and fricatives are phonemically unspecified for voice, with sonorants more often surfacing as voiced and fricatives more often surfacing as unvoiced.

\subsection{Vowels}\label{ssec:vowel_inv}

\begin{wraptable}[7]{r}{.35\linewidth}
    \centering
    \begin{tabular}{@{}rccc@{}}
    \toprule
        & Front & Back  & Rounded \\ \midrule
    High & i     & \unru & u       \\
    Mid  & e     & \unro & o       \\ \bottomrule
    \end{tabular}
    \caption{Vowel Inventory}
    \label{tab:vowels}
\end{wraptable}

By contrast with its large consonant inventory, \lang{}'s vowel inventory is reasonably small, with six vowels. Most strikingly, \lang{} lacks any vowels lower than mid-close, resulting in an apparently very lopsided inventory. Some have speculated that this is the case because goblins mouths are naturally shorter or more close than the mouths of other species, but this has yet to be tested empirically.

\begin{wrapfigure}[11]{l}{.35\linewidth}
    \centering
    \footnotesize
    \begin{forest}
        [{[high]}
            [{[back]}, edge label={node[midway,above left,font=\scriptsize]{+}}
                [{\unru}, edge label={node[midway,above left,font=\scriptsize]{+}}]
                [{[rounded]}, edge label={node[midway,fill=white,font=\scriptsize]{±}}
                    [{u}, edge label={node[midway,fill=white,font=\scriptsize]{+}}]
                ]
                [{i}, edge label={node[midway,above right,font=\scriptsize]{--}}]
            ]
            [{[back]}, edge label={node[midway,above right,font=\scriptsize]{--}}
                [{\unro}, edge label={node[midway,above left,font=\scriptsize]{+}}]
                [{[rounded]}, edge label={node[midway,fill=white,font=\scriptsize]{±}}
                    [{o}, edge label={node[midway,fill=white,font=\scriptsize]{+}}]
                ]
                [{e}, edge label={node[midway,above right,font=\scriptsize]{--}}]
            ]
        ]
    \end{forest}
    \caption{Contrastive Hierarchy of \lang{} Consonants}
    \label{fig:vowel_hierarchy}
\end{wrapfigure}

\lang{} vowels can be neatly divided into three classes: front, back (unrounded), and (back) rounded. As indicated by the parentheses in the prior sentence, the unroundedness of the back unrounded vowels and the backness of the back rounded vowels are incidental and not phonemically meaningful. As a result, the back rounded vowels never trigger backing as the back unrounded vowels do, and the back unrounded vowels have no effect on rounding, unlike the back rounded vowels.

These classes participate in \lang{}'s system of vowel harmony, in which the leftmost (first) vowel occurring in a phonological word conditions the phonetic realizations of the following vowels. If the first vowel in the phonological word is front, all following vowels in the same phonological word are fronted. If the first vowel in the phonological word is back (unrounded), all following vowels in the same phonological word are backed. If the first vowel in the phonological word is (back) rounded, all following vowels in the same phonological word are rounded. More detail about the exact forms realized can be seen in \autoref{tab:vowel_harm}

\begin{wraptable}[13]{r}{.35\linewidth}
    \centering
    \begin{tabular}{@{}rccc@{}}
    \toprule
                      & Front         & Back          & Rounded   \\\midrule
        \phipa{i}     & \bripa{i}     & \bripa{\bari} & \bripa{y} \\
        \phipa{e}     & \bripa{e}     & \bripa{\sche} & \bripa{ø} \\
        \phipa{\unru} & \bripa{\bari} & \bripa{\unru} & \bripa{u} \\
        \phipa{\unro} & \bripa{\sche} & \bripa{\unro} & \bripa{o} \\
        \phipa{u}     & \bripa{y}     & \bripa{u}     & \bripa{u} \\
        \phipa{o}     & \bripa{ø}     & \bripa{o}     & \bripa{o} \\
        \bottomrule
    \end{tabular}
    \caption{Realizations of phonemic vowels in different vowel harmony environments}
    \label{tab:vowel_harm}
\end{wraptable}

It is noteworthy that, although both the phonemically back vowels and the phonemically rounded vowels are realized as \bripa{u} and \bripa{o} according to this table, speakers generally have no trouble distinguishing between them in context. Explanations for this have varied, with some theorizing that the phonemically rounded vowels are slightly further front than their phonemically backed counterparts and others arguing that there are more complex differences in the articulations that cannot be described as simply frontness or closeness. Yet others argue that this is solely the influence of the surrounding context and that in isolation they would not be so easily distinguished. More research is necessary to determine which of these groups is in the right.


\section{Syllable Structure}

While there are two competing analyses of \lang{} syllable structure, distinguished by how they treat \lang{}'s syllabic consonant nuclei on a phonemic level. The dominant analysis is the so-called `vowelless' analysis, which analyzes the nucleus as composed of a vowel and any non-stop (that is, sonorant or continuant, represented by R in the figures), both of which are optional but at least one of which must be present. The remaining rime consists of an optional coda consisting of any single consonant. 

\begin{wrapfigure}{r}{0.35\textwidth}
    \centering
    % \begin{tikzpicture}
    %     \Tree [.$\sigma$ [.$\omega$ S* C* R* ] [.$\rho$ [.$\nu^{+}$ V* R* ] [.$\kappa$ C* ] ] ]
    % \end{tikzpicture}
    \begin{forest}
        [\(\sigma\),  for tree={parent anchor=south, child anchor=north}
            [\(\omega\)
                [S*]
                [C*]
                [R*]
            ]
            [\(\rho\)
                [\(\nu^{+}\)
                    [V*]
                    [R*]
                ]
                [\(\kappa\)
                    [C*]
                ]
            ]
        ]
    \end{forest}
    \caption{Syllable Structure in Vowelless Analysis}
    \label{fig:sylb-struc-vless}
\end{wrapfigure}

The onset is also optional, at least on a phonemic level, and can consist of any consonant, optionally preceded by a sibilant fricative (represented by S in the figures) and/or followed by a non-stop (that is, sonorant or continuant, represented by R in these diagrams). However, as \lang{} does not permit vowel hiatus, the onset is only optional word-initially or when following a syllable with a coda. A syllable appearing after a coda-less syllable requires an onset to avoid hiatus.

This syllable structure can also be expressed in Recursive Baerian Phonotactic Notation
\footnote{Described in \url{https://llblumire.github.io/recursive-baerian-phonotactics-notation/Recursive_Baerian_Syntax_Notation.pdf}. Here we make some minor changes to the notation. Simple \(\begin{matrix}\textrm{A}\\\nm\end{matrix}\) stacks are rendered as \((\textrm{A})\) for the sake of simplicity and brevity. The labels for the blocks have been moved for readability, now subscripted after the final bracket. We also use $\lambda$ for word-level blocks such that $\omega$ can be freed to be used for the syllable-onset block, in keeping with traditional syllable notation.}%
as follows:

\begin{equation*}
    \#
    \left[%
        \begin{matrix}
            {%
                \left[%
                    (\textrm{S}) \: \textrm{C} \: (\textrm{R})
                \right]_\omega%
            }%
            \\
            \nm
        \end{matrix}
        \left[%
            \left[\:%
                \begin{matrix}
                    {\textrm{V} \: (\textrm{R})}
                    \\
                    \textrm{R}
                \end{matrix}
            \right]_\nu%
            \:%
            \begin{matrix}
                \textrm{C} & 
                {
                    \begin{matrix}
                        (\omega)\:\rho
                        \\
                        \nm
                    \end{matrix}
                }
                \\
                \nm & 
                {
                    \begin{matrix}
                        \omega\:\rho
                        \\
                        \nm
                    \end{matrix}
                }
            \end{matrix}
        \right]_\rho%
    \right]_\lambda%
    \#
\end{equation*}
% \begin{equation*}
% \# \left[{\color{cyan}\begin{matrix}{\left[\:\textcolor{black}{(\textrm{S}) \: \textrm{C} \: (\textrm{R})}\right]_\omega}\\\nm\end{matrix}}{\color{magenta}\left[\:\textcolor{black}{\begin{matrix}\:\textrm{V} \: (\textrm{R})\\\textrm{R}\end{matrix}\begin{matrix}\textrm{C} \: \begin{matrix} \textcolor{cyan}{(\omega)} \, \textcolor{magenta}{\rho} \\ \nm\end{matrix}\\\nm \: \begin{matrix} \textcolor{cyan}{\omega} \, \textcolor{magenta}{\rho} \\ \nm\end{matrix}\end{matrix}}\right]_\rho}\:\right]_\lambda\#
% \end{equation*}

\subsection{Underlying Vowel Analysis}

\begin{wrapfigure}[14]{L}{0.35\textwidth}
    \centering
    \begin{forest}
        [\(\sigma\),  for tree={parent anchor=south, child anchor=north}
            [\(\omega\)
                [S*]
                [C*]
                [R*]
            ]
            [\(\rho\)
                [\(\nu\)
                    [V\(^{+}\)]
                ]
                [\(\kappa\)
                    [R*]
                    [C*]
                ]
            ]
        ]
    \end{forest}
    \caption{Syllable Structure in Vowelless Analysis}
    \label{fig:sylb-struc-schwa}
\end{wrapfigure}

Among those who don't subscribe to an analysis of \lang{} as possessing vowelless syllables, the dominant approach is to analyze the syllable structure as underlyingly more symmetric and regular, such as that shown in \autoref{fig:sylb-struc-schwa}, with a mandatory vowel nucleus and both onset and coda allowing any consonant to cluster with a non-stop. Within this analysis, what phonetically surface as syllabic consonants are really just allophonous realizations of an underlying unrealized vowel + consonant clusters. 

In Baerian notation this is structured identically to the vowelless analysis, except that the \(\nu\) block is replaced with a simple \(\textrm{V}\:(\textrm{R})\), with allophony and an underlying seventh vowel accounting for realization of the nucleus as a syllabic consonant phonetically in some contexts.

Those who champion this underlying vowel analysis tend to point to the relative symmetry of this structure compared to the more complicated analysis that allows for vowelless nuclei. However, those who oppose this analysis tend to consider the lack of similarity to surface forms as a point against this analysis. 

In practice, both analyses generally can capture \lang{} phenomena. In this grammar, we will point out where differences between these two analyses become relevant and refer to both analyses in those cases, but by and large we will focus on the actual changes in the surface forms rather than on their theoretical syllabic structure.

\section{Phonotactics \& Allophony}\label{sec:allophony}



\subsection{Consonant Harmony}\label{ssec:consonant_harmony}

In addition to vowel harmony, \lang{} also possesses consonant harmony involving hissing vs. hushing sibilants. Unlike vowel harmony, this harmony is triggered by the rightmost occurrence of a relevant segment within a phonological word. If the last coronal sibilant in a phonological word is post-alveolar, all preceding sibilants within that phonological word will also be post-alveolar. `Laminarity' is retained, so denti-alveolar sibilants become palatoalveolar while apicoalveolar sibilants become alveolopalatal. If the last coronal sibilant is alveolar, it similarly causes all preceding sibilants within that phonological word to be alveolar rather than post-alveolar, maintaining laminarity where possible.

While this consonant harmony is only triggered by sibilants, it also affects liquids, causing \phipa{l\lamino} and \phipa{l\apico} to pattern with the hissing sibilants and \phipa{r\bck} and \phipa{r\bck\pal} to pattern with the hushing sibilants. However, the liquids never trigger this harmony and only conform to it if sibilants are present later on in the word, so they can co-occur with the `wrong' sibilant under some circumstances.

\subsection{Laminal Assimilation}


\begin{enumerate}
    \item Coronal consonants persistently assimilate in laminarity to the following consonant.
    \begin{itemize}
        \item syllabic consonants included
        \item reflected in orthography
    \end{itemize}
    \item Sibilants become alveolopalatal after a palatal \& palatal obstruents become alveolopalatal after a sibilant
    \begin{itemize}
        \item syllabic consonants included
        \item \emph{not} reflected in orthography
    \end{itemize}
    \item Palatals become velar when they precede a non-palatal consonant
    \begin{itemize}
        \item syllabic consonants \emph{not} included
        \item \emph{not} reflected in orthography
    \end{itemize}
    \item Velars and non-sibilant coronals are palatalized before a palatal consonant
    \begin{itemize}
        \item syllabic consonants included
        \item \emph{not} reflected in orthography
    \end{itemize}
    \item Obstruents persistently assimilate in voice to a following obstruent. Approximants devoice before unvoiced obstruents.
    \begin{itemize}
        \item syllabic consonants included
        \item reflected in orthography \emph{sometimes} (i.e., when there is a symbol for the voiced/devoiced version of the respective sound)
    \end{itemize}
    \item Nasals assimilate in POA to a following plosive or nasal
    \begin{itemize}
        \item syllabic consonants \emph{not} included
        \item reflected in orthography
    \end{itemize}
    \item Nasals become partially-voiced non-nasal stops before a voiceless obstruent
    \begin{itemize}
        \item only applies to non-syllabic nasals
        \item only reflected in orthography when it occurs due to productive \emph{derivational} morphology, not when it occurs within a root or due to inflection
    \end{itemize}
    \item Coronal obstruents assimilate in sibilancy to a following fricative. Before a non-sibilant fricative, sibilant obstruents become lateral. Before a sibilant fricative, non-affricate stops become affricates
    \begin{itemize}
        \item syllabic consonants \emph{not} included
        \item reflected in orthography
    \end{itemize}
    \item Two of the same consonant in a row merge into a single occurrence of that consonant. Fricatives following affricates at the same place of articulation are also deleted. Partially-voiced stops absorb following voiceless stops at the same place of articulation.
    \begin{itemize}
        \item syllabic consonants included
        \item reflected in orthography
    \end{itemize}
    \item Lateral fricatives cannot be adjacent to lateral approximants
    \begin{itemize}
        \item if the approximant's placement violates the sonority hierarchy (if they are in the same syllable and it occurs before the fricative in the onset or after the fricative in the coda), the approximant is deleted
        \item if they meet at a syllable boundary, they merge and the resulting phone takes on the voicing of the latter and the manner of articulation of the former
        \item otherwise, the lateral fricative is realized as a velar non-lateral fricative
        \item reflected in orthography, but 
        %\bripa{\latfrivoic} and \bripa{l\vless} are still represented as clusters
    \end{itemize}
\end{enumerate}

\section{Romanization}



\begin{center}
    
    \begin{tabular}{cc}
        \toprule
        Roman. & Phon. \\\midrule
        %coronals
        %apical
        \ortho{t} & \bripa{t\apico}\\
        \ortho{d} & \bripa{d\apico}\\
        \ortho{c} & \bripa{t\tiebar s\apico}\\
        \ortho{z} & \bripa{d\tiebar z\apico}\\
        \ortho{s} & \bripa{s\apico}\\
        \ortho{n} & \bripa{n\apico}\\
        \ortho{\l} & \bripa{\latfric\apico}\\
        \ortho{l} & \bripa{l\apico}\\
        %laminal
        \ortho{ty} & \bripa{t\lamino}\\
        \ortho{dy} & \bripa{d\lamino}\\
        \ortho{cy} & \bripa{t\tiebar s\lamino}\\
        \ortho{zy} & \bripa{d\tiebar z\lamino}\\
        \ortho{sy} & \bripa{s\lamino}\\
        \ortho{ny} & \bripa{n\lamino}\\
        \ortho{\l y} & \bripa{\latfric\lamino}\\
        \ortho{ly} & \bripa{l\lamino}\\
        \bottomrule
    \end{tabular}
    \hspace{1em}
    \begin{tabular}{cc}
        \toprule
        Roman. & Phon. \\\midrule
        %pal-alv
        \ortho{ch} & \bripa{t\tiebar\esh}\\
        \ortho{j} & \bripa{d\tiebar\ezh}\\
        \ortho{sh} & \bripa{\esh}\\
        \ortho{r} & \bripa{\alvr\bck\apico}\\
        %alv-pal
        \ortho{chy} & \bripa{t\tiebar\alvpalesh}\\
        \ortho{jy} & \bripa{d\tiebar\alvpalezh}\\
        \ortho{shy} & \bripa{\alvpalesh}\\
        \ortho{ry} & \bripa{\alvr\bck\apico\pal}\\
        %labials
        \ortho{p} & \bripa{p}\\
        \ortho{b} & \bripa{b}\\
        \ortho{m} & \bripa{m}\\
        %back consonants
        \ortho{k} & \bripa{k}\\
        \ortho{g} & \bripa{g}\\
        \ortho{\engma} & \bripa{\engma}\\
        \ortho{\okina} & \bripa{\glotstop}\\
        \bottomrule
    \end{tabular}
    \hspace{1em}
    \begin{tabular}{cc}
        \toprule
        Roman. & Phon. \\\midrule
        \ortho{ee} & \bripa{i}\\
        \ortho{e} & \bripa{e}\\
        \ortho{eu} & \bripa{\unru}\\
        \ortho{eo} & \bripa{\unro}\\
        \ortho{oo} & \bripa{u}\\
        \ortho{o} & \bripa{o}\\
        \ortho{iu} & \bripa{y}\\
        \ortho{io} & \bripa{ø}\\
        \ortho{i} & \bripa{\bari}\\
        \ortho{u} & \bripa{\sche}\\
        \ortho{a} & \nm\\
        \bottomrule
    \end{tabular}
\end{center}

When multiple laminal consonants occur in the same cluster, only a single \ortho{y} at the end of the cluster is used.

\chapter{Morphology}

\section{Nominal Morphology}

\subsection{Noun Class \& Agreement}

\subsubsection{Classes}

\lang{} has roughly five to seven noun classes, depending on how one analyzes the behavior of plural nouns. As a compromise, the `plural classes' are often indicated as sub-classes of the appropriate noun class, which is the approach taken here. These classes are \emph{largely} determined by the semantics of the lexical item, but there are often apparent exceptions and strange behavior according to outsiders.

\paragraph{Class I: Dominant Forces}

Perhaps the most apparently eclectic of the \lang{} noun classes, this class contains nouns that refer to anything markedly more powerful than a goblin, to the extent that an individual goblin doesn't have much hope against it on its own. This includes natural forces, such as celestial bodies, weather events, fire, explosions, etc., but it also includes most non-goblin humanoids, mythical beasts, particularly dangerous or enormous animals, massive moving structures like ships, and later powerful weapons like cannons. This class is even used to refer to some goblins when those goblins have a particularly high degree of power or skill to the extent that they become threatening enough for this class. In this case, the noun class assignment ends up serving as a sort of honorific.

\paragraph{Class I-a: Collectives}

One of the `plural classes', this class encompasses large groups of animate creatures acting as one. `Animate' here is used rather loosely, referring to anything that can act on its own power. Since \lang{} pluralization is solely derivational, the plural forms of many nouns from classes II-III and even some from other classes are sorted into this class and are treated like Class I nouns in terms of agreement. For instance, the plural of the class II noun \emph{geob} `goblin', \emph{geogeob} `a large group of goblins' is a class I-a noun.

\paragraph{Class II: Goblins}

This class is fairly self-explanatory---it includes goblins, parts of goblins, and anything else sufficiently intelligent but non-threatening enough to be considered sufficiently goblin-like by a goblin. This generally does not include so-called `tall people', who are usually placed into Class I, but goblins have been heard referring to elf or human children who are old enough to talk and communicate but young enough to not pose a threat using this class.

\paragraph{Class III: Animals}

Any creature that not intelligent enough for speech and weak enough that a single goblin could potentially at least defend itself against it successfully is sorted into this noun class. This includes many wild animals, but also babies even of intelligent species. Whether a dead animal is considered part of this class, however, depends on the context. Portions of a dead animal that are to be used for food or other purposes tend to be re-categorized into Class IV or V, but corpses of creatures that were particularly beloved or which had the potential of becoming intelligence (such as infants) tend to stay in this class.

\paragraph{Class IV: Quasi-Animate Things}

This class contains things that aren't considered fully animate, since they cannot act on their own will, but which can have strong effects (positive or negative) on the world if used by a truly animate being. Most tools fall into this category, as do most plants and their parts, as well as any particularly potent consumables such as medicine or drugs. 

\paragraph{Class V: Inanimate Objects}

This class consist of the remainder of inanimate objects---anything which is part of the physical world but which doesn't have a strong enough effect on the world even in the hands of others to be classified as class IV. 

\paragraph{Class VI: One-in-Many Objects}

By far the smallest of the noun classes, this class is often ignored as being purely the product of plural inflection, as it contains the derivational plurals of several words that belong to Classes IV and V. This class consists of inanimate objects that are composed of a multitude of objects, but which are not inherently threatening enough for Class I-a. For instance, the word \emph{shyopa} `sand' (which, incidentally, is not derived from any other noun) is in Class VI, while the word \emph{keershyiope} `sandstorm' is in Class I.

\subsubsection{Agreement}\label{sssec:noun_class_agreement}

Agreement with noun class occurs in several places in \lang{} grammar. Different elements of the grammar treat this agreement somewhat differently, however, and distinguishing these differences is important for grammatical agreement in \lang{}. Three major classes of elements in \lang{} grammar agree with the class of some noun: verbs and postpositions, nominal possessives, and pronouns. Nominal possessives are discussed further in \autoref{ssec:possession}, while the other two classes are discussed here.

\paragraph{Verbs \& Postpositions}

Experts disagree on whether verbs and postpositions can be considered truly separate classes in \lang{}, but regardless of which conclusion one draws in that debate, it is clear that \lang{} verbs and postpositions use the same mechanisms for agreement. Both are inflected with suffixes to agree with their objects in person and, in the case of third-person objects, in noun class.

\begin{center}
    \begin{tabular}{cccccccc}
        \First & \Second & \Third.\Dom & \Third.\Gob & \Third.\Anim & \Third.\Use & \Third.\Inan & \Third.\Many\\
        \textit{-edy} & \textit{-(a)rk} & \textit{-meu} & \textit{-(eo)g} & \textit{-sy(u)} & \textit{-(o)n} & \textit{-(e)s} & \textit{-eem}
    \end{tabular}
\end{center}

As they are suffixes, naturally the forms of these agreement markers can differ in accordance with vowel harmony. Those forms with initial vowels in parentheses indicate that those vowels only occur when the word they are suffixed to ends in a consonant but do not occur when the word they are suffixed to ends in a vowel. The forms with initial vowels that are not in parentheses replace the final vowel of any vowel-final word they are suffixed to with the initial vowel of the suffix.


\paragraph{Pronouns}

Naturally, pronouns agree with noun class with their referents. However, unlike the agreement markers on verbs and postpositions, pronouns do not agree with the person of their referent; there is no distinction between a first-person reference to a goblin and a third-person reference to a goblin, at least when it comes to pronominal agreement. 

\begin{center}
    \begin{tabular}{rcccccc}
        & \Dom & \Gob & \Anim & \Use & \Inan & \Many \\
        Unmarked & \textit{meum} & \textit{eog} & \textit{syu} & \textit{nyon} & \textit{es} & \textit{\okina eem}\\
        Emphatic & \textit{ammeu} & \textit{geog} & \textit{syusy} & \textit{nyonan} & \textit{ses} & \textit{\okina ee\engma eem}
    \end{tabular}
\end{center}

    

Disambiguation of referents based on person can rely either on the agreement markers on adjacent verbs and postpositions, but speakers also often use names and other nominal repetition for disambiguation more often than speakers of many other languages.

\subsection{Pluralization}

What is known as the plural in \lang{} is a nominal form created by reduplicating the onset and nucleus of the first syllable of the word with an onset. For instance, the word \emph{geob} `goblin' is pluralized into \emph{geogeob} `a multitude of goblins', while \emph{emutee} `elf' is pluralized into \emph{memutee} `a multitude of elves'.

\lang{} pluralization is an entirely optional process that is best analyzed as derivational process rather than an inflectional one---that is, the pluralization creates a new word whose semantics roughly correspond to a multitude of the noun from which it's derived, rather than being simply a grammatically plural form of the original noun. There is a fairly large amount of evidence for this, which will go here once sparky feels like working on this section more.

Not every noun in \lang{} can be pluralized, and indeed, pluralization is generally used for something more akin to emphasis of the large quantity of the underlying noun as opposed to simple pluralization. Several nouns, generally those in classes I(a) and VI, inherently refer to a collective or large quantity of something and instead have ways to derive words referring to their smallest components. This will also be elaborated on in this section when sparky feels like finishing this.

\subsection{Possession}\label{ssec:possession}

Possession in \lang{} is marked on the possessee, as \lang{} has a strong preference for head-marking structures. How the possessee is marked as possessed varies depending on certain lexical and morphosyntactic factors.

Certain nouns are obligatorily possessed and do not exist in isolation without any possessive morphology. They surface only with one of a variety of possessive prefixes, which agree with the possessor in person and, in the case of third-person possessors, noun class.

\begin{center}
    \begin{tabular}{cccccccc}
        \First & \Second & \Third.\Dom & \Third.\Gob & \Third.\Anim & \Third.\Use & \Third.\Inan & \Third.\Many\\
        \textit{dye(t)-} & \textit{k(a)r-} & \textit{meu(m)-} & \textit{eog(eo)-} & \textit{syu(sy)-} & \textit{nyo(n)-} & \textit{s(e)-} & \textit{\okina ee(\engma)-}
    \end{tabular}
\end{center}

\noindent As they generally occur in the leftmost position, these prefixes can and do affect the vowel harmony of the rest of the word, and are likewise affected by the rest of the word in respect to consonant harmony.
\newpage
To give an example, the \lang{} word for `mother' \textit{-meo} is obligatorily possessed (like, in fact, most terms referring to familial relationships). \textit{-meo} never surfaces in isolation and only occurs with one of these prefixes: \textit{dyemu} `my mother', \textit{karmeo} `your mother', \textit{eogmeo} `their mother', etc. Obligatorily possessed words tend to be words who tend to involve some form of lexically-inherent inalienable possession, such as body parts and family members.

Possessive inflection for standalone nouns involves addition of the suffix \textit{-(a)\l}, which derivationally forms a postposition meaning `possessing [noun]'. This is then inflected just as any other postposition, with the possessor serving as the object of the postposition. \textsc{poss} \textsc{foss} \textsc{ooss}

\ex
\begingl
kar[\Second.\Poss]@
-meo[-mother]
jryeshy[rucksack]@
-a\l y[-\Poss]@
-ug[-\ToGob]
\glft
``Your mother's rucksack''
\endgl
\xe

This resulting postpositional phrase can still be used as an argument, as technically all postpositional phrases can, but like other postpositional phrases, verbal agreement is with the \emph{object} of the preposition, not the noun from which the postposition is derived.

\pex
\a
\begingl
jryeshy[rucksack(\Inan)]
es oom[there.is]@
\textbf{-ios}[\textbf{-\ToInan}]
\glft
``There's a rucksack.''
\endgl
\a
\begingl
karmeo[your.mother(\Gob)]
jryeshya\l yug[her.rucksack]
es oom[there.is]@
\textbf{-og}[\textbf{-\ToGob}]
\glft
``There's your mother's rucksack.''
\endgl
\xe

These possessive derived postpositions are not used exclusively for what we would traditionally consider possessives and are often used for purposes most would consider somewhat orthogonal to possession, such as quantification. Such further uses of this construction are discussed in \autoref{chap:semantics}.

\subsection{Other Derivational Operations}

\section{Verbal Morphology}

\chapter{Syntax}\label{chap:syntax}

\chapter{Semantics}\label{chap:semantics}

\setsecnumdepth{part}
\settocdepth{part}

%\chapter{t ť d ď c č z ž s š n ň r ř l ľ ł \dbl{} λ ƛ p b m ć ź ń k g ŋ h x i e u o eu eo a}

\begin{multicols*}{2}

\section{T}



\end{multicols*}

\end{document}




\section{Verbal Morphology}


\subsection{Animacy Hierarchy}

Geobo{\engma}'s parent language (known as \textit{Cokizbo\engma}, lit. `finished language', to the goblins) was a direct-inverse language, and as such had an animacy hierarchy among what are now the noun classes. Modern Geobo{\engma} only shows remnants of this hierarchy in the verbal morphology.

\begin{table}[h]
    \centering
    \begin{tabular}{@{}rl@{}}
    \toprule
0 & 1st \& 2nd Person \\
1 & Dominant Forces \\
2 & Goblins \\
3 & Animals \\
4 & Quasi-animate/Useful Things \\
5 & Inanimate Objects \\
6 & Abstract Concepts \\
\bottomrule
\end{tabular}
    \caption{Animacy Hierarchy}
    \label{tab:anim}
\end{table}


\subsection{Verbal Agreement}

Verbs in Geoboŋ are marked with a suffix that agrees with the syntactic object in person/noun class. These markers originate from pronouns that used to be placed after the verb and thus resemble reduced forms of those pronouns. 

% \begin{table}[ht]
%     \centering
%     \begin{threeparttable}
%     \begin{tabular}{@{}rcccccccc@{}}
% \toprule
%  & 1st & 2nd & \textsc{dom} & \textsc{gob} & \textsc{anim} & \textsc{quas} & \textsc{inan} & \textsc{abst} \\ \midrule
% Suffix\tnote{1} & \textit{-e\vd, -\vd} & \textit{-ar, -r} & \textit{-meu} & \textit{-eog, -g}\tnote{2} & \textit{-\vs u} & \textit{-a\vn, -\vn} & \textit{-im, -m}\tnote{3} & \textit{-es, -s}\tnote{4}  \\
% Pronoun & \textit{te\vd} & \textit{kar} & \textit{ameu} & \textit{eog} & \textit{\vs u} & \textit{\vn o\vn} & \textit{him} & \textit{es} \\
% Reflexive & \textit{tete\vd} & \textit{karkar} & \textit{meum} & \textit{geog} & \textit{\vs u\vs} & \textit{\vn o\vn a\vn} & \textit{hiŋim} & \textit{ses} \\ \bottomrule
% \end{tabular}
% \begin{tablenotes}
%     \item[1] When multiple suffixes are listed, the former is used when the suffix follows a non-syllabic consonant and the latter when it follows a syllabic consonant or vowel unless otherwise indicated.
%     \item[2] If the latter suffix is used following a front vowel, that vowel is backed.
%     \item[3] If the latter suffix is used following a back vowel, that vowel is fronted. Fronting of rounded vowels is allophonic and not reflected orthographically.
%     \item[4] Unlike the other listed suffixes, for the abstract noun class the latter vowelless allomorph is only used following a true vowel, and the former suffix is used following a syllabic consonant.
% \end{tablenotes}
% \end{threeparttable}
%     \caption{Person/Animacy Markers}
%     \label{tab:my_label}
% \end{table}

These markers agree in person and animacy with the syntactic \emph{object} rather than the subject---that is, generally the more patientlike P argument rather than the more agentlike A argument (though the semantic roles these syntactic roles fill can obviously vary significantly depending on both the verb in question and the context). The reflexive pronouns shown here occur in the subject position, never in the object position, when the subject and object are the same entity.

If both the subject and object are of the same noun class and it is thus ambiguous which entity in the sentence the agreement marking is referring to, the appropriate standalone pronoun can be placed directly after the subject. However, this is only done if the sentence would be obviously ambiguous otherwise.

% \pex
% \a
% \begingl
% \glpreamble
% `Yesterday, I saw the dog and the cat fighting, and then...
% \endpreamble
% de{\vz}[cat]
% a\l k[dog]
% \vs u[\sc 3.anim]
% bo\vl[strike]@
% -\vs u[\sc -3.anim]
% \glft ...the dog hit the cat.'
% \endgl
% \a 
% \begingl
% \glpreamble
% `I told my dog to stay, but then when I turned around...
% \endpreamble
% de{\vz}[cat]
% a\l k[dog]
% bo\vl[strike]@
% -\vs u[\sc -3.anim]
% \glft ...the dog hit the cat.'
% \endgl
% \xe

\section{Verbs \& Transitivity}

Agreement with the syntactic P argument naturally leads to the question of what (if any) markings verbs take if the verb is intransitive. The simple answer is that Geoboŋ lacks true syntactic intransitives entirely. Rather, several different constructions are used as workarounds so that otherwise transitive Geoboŋ verbs can express typically-intransitive meanings.

For verbs in which the single argument does not initiate or is not actively responsible for the action of the verb, the most common solution is to use the abstract pronoun \textit{es} as a dummy subject, similar to how English would use `there' or `it' in certain types of clauses. 

\ex 
\begingl
Geob[goblin]
es[\sc 3.abst] 
\l emp[trip]@
-eog[\sc -3.gob]
\glft `The goblin trips.' (lit., `It trips the goblin.')
\endgl
\xe

There does not have to be an unspoken agent or cause of the action described by the verb for this structure to work---the purest of statives with no actual cause can also be represented by this structure.

% \ex
% \begingl
% Beu[berry]
% es[\sc 3.abst]
% ka\l[red]@
% -a\vn[\sc -3.quas]
% \glft `The berry is red.' (lit., `It reddens the berry.')
% \endgl
% \xe

Many verbs, particularly those of position and motion, contrast this dummy-subject structure with another that uses reflexive pronouns. For these verbs, use of a reflexive pronoun as the subject indicates a dynamic reading, while the dummy-subject structure indicates a stative reading.

\pex
\a
\begingl
Geob[goblin]
es[\sc 3.abst]
ot[sit]@
-eog[\sc -3.gob]
\glft `The goblin is sitting.' (lit., `It seats the goblin.')
\endgl
\a
\begingl
Geob[goblin]
geog[\sc 3.gob.refl]
ot[sit]@
-eog[\sc -3.gob]
\glft `The goblin sits down.' (lit., `Himself seats the goblin.')
\endgl
\xe

As evidenced by the use of the dummy-subject strategy for a verb like `to sit', it doesn't take much for an argument to be considered sufficiently un-agent-like for that strategy. Indeed, only the most prototypical agents are not captured by this strategy.

For would-be intransitive verbs where the argument explicitly represents an agent (known as unergative verbs in English grammar), there is yet another strategy. One can nominalize the main verb adding the nominalizing suffix \textit{-stu} to the root, and then treat this nominalized verb as the object of the verb \textit{\'se} `to do, to make'.

% \ex
% \begingl
% \vR im[run]@
% -stu[\sc -nmz]
% geob[goblin]
% \'se[do]@
% -s[\sc -3.abst]
% \glft  `The goblin is running.' (lit., `The goblin does running.')
% \endgl
% \xe

For certain motion verbs, both the nominalization and reflexive strategies are both appropriate. In this case, the former allows the speaker to put the action in focus in a way that cannot be done when the action is not nominalized, as in the reflexive strategy (see section \ref{focus} for more on focus marking). In addition, the former carries connotations of duration/ongoingness, while the latter is more explicitly inchoative. Compare (\lastx) with (\nextx):

% \ex
% \begingl
% Geob[goblin]
% geog[\sc 3.gob.refl]
% \vr im[run]@
% -eog[\sc -3.gob]
% \glft `The goblin gets himself running.'
% \endgl
% \xe

The reflexive strategy is not necessarily appropriate for all would-be intransitives, however, as for many (if not most) verbs, it simply carries the expected reflexive meaning.

\ex
\begingl
Karkar[\sc 2.refl]
bol[strike]@
-ar[\sc -2]
\glft  `You're hitting yourself.' (\emph{not} `You're getting yourself to hit things.')
\endgl
\xe

\section{Possession \& Quantification}

As Geobo{\engma} is head-marking, in a possession scenario it is the possessed noun that is inflected. A possessive affix \textit{-(a)\l-} is added to the end of the possessed noun and then followed by a suffix that agrees in person and number with its possessor. The possessor, if specified, is placed before the possessed noun.

% \ex
% \begingl
% \vN u\'noh[elf]
% meo[mother]@
% -\l[\sc -poss]@
% -meu[\sc -3.dom]
% \glft  `The elf's mother'
% \endgl
% \xe

This syntactic structure is also used more broadly, however, for quantification. Perhaps counterintuitively, the quantified noun is treated as the possessor of the noun indicating the relevant quantity.

% \ex
% \begingl
% \vD eo\'s[tree]
% eu\'s[all]@
% -a\dbl[\sc -poss]@
% -\vn[\sc -3.quas]
% \glft `all of the trees' (lit., `the tree's everything')
% \endgl
% \xe

This is not limited to more ambiguous quantifiers---it is used for numbers as well.

% \ex
% \begingl
% \vD eo\'s[tree]
% zlic[three]@
% -a\dbl[\sc -poss]@
% -\vn[\sc -3.quas]
% \glft `three trees' (lit., `the tree's three')
% \endgl
% \xe

This use of quantifiers is one of Geobo{\engma}'s main methods of compensating for the fact that it does not mark for number whatsoever, even in pronouns. If a speaker wants to pluralize a pronoun (for instance, saying `we') that speaker can simply use the quantifier inflected for the relevant person and animacy 
%(such as \textit{eu\'sa\dbl\vd} `all of me/us').

To quantify over verbs (i.e., to say something happened `twice' or `many times'), one must quantify over the object. This is true even if the object seems to be something unquantifiable, such as an abstract noun or nominalized verb.

% \ex
% \begingl
% \vR im[run]@
% -stu[\sc -nmz]
% kle[two]@
% -\l[\sc -poss]@
% -es[\sc -3.abst]
% te{\vd}[1]
% \'se[do]@
% -s[\sc -3.abst]
% \glft `I ran twice.' (lit., `I did two running.')
% \endgl
% \xe

Distinguishing between whether a quantifier on the object semantically indicates quantification over the object or the verb is only done if context does not make it clear which is meant, but can be done using adjectives and embedding quantifiers under one another.

% \pex
% \a
% \begingl
% \vD er\engma[different]
% \vs\vr eo\vd[eel]
% kle[two]@
% -\dbl[-{\sc poss}]@
% -\vs u[\sc -3.anim]
% te\vd[1]
% \vr u[cook]@
% -\vs u[\sc -3.anim]
% \glft `I cooked two different eels.'
% \endgl
% \a
% \begingl
% \vS\vr eo\vd[eel]
% da\'n[one]@
% -a\dbl[\sc -poss]@
% -\vs u[\sc -3.anim]
% kle[two]@
% -\l[\sc -poss]@
% -es[\sc -3.abst]
% te\vd[1]
% \vr u[cook]@
% -\vs u[\sc -3.anim]
% \glft `I cooked one eel twice.'
% \endgl
% \xe

\section{Postpositions}

Postpositions follow their nouns and inflect to agree with their objects. Their endings are identical to those used for verbs.

% \ex
% \begingl
% \vT\'ne{\vl}[stick]
% \vc o[\sc ins]@
% -m[\sc -3.inan]
% \glft `with a stick'
% \endgl
% \xe

If the postposition would only be preceded by a pronoun, it can be omitted and the inflected postposition used on its own as a whole phrase.

% \ex 
% \begingl 
% Ŋi[away.from]@
% -\vd[-1]
% \vn u\'noh[elf]
% meum[\sc 3.dom.refl]
% \vr im[run]@
% -eu[\sc -3.dom]
% \glft `The elf ran away from me.'
% \endgl
% \xe

In addition to traditional postpositions, Geobo{\engma} demonstratives are also realized as postpositions. Since postpositional phrases cannot be embedded within each other, one cannot use an explicit demonstrative with a noun that is already part of a postpositional phrase.

% \pex
% \a
% \begingl
% \Engma u{\vz}[fish]
% u\vt[\sc med]@
% -\vs u[\sc -3.anim]
% kar[2]
% bo\vs[eat]@
% -\vs u[\sc -3.anim]
% \glft `You ate that fish.'
% \endgl
% \a 
% \begingl 
% \Engma u\vz[fish]
% \vc o[\sc ins]@
% -\vs u[\sc -3.anim]
% kar[2]
% bol[strike]@
% -e\vd[\sc -1]
% \glft `You hit me with a fish.'
% \endgl
% \a \ljudge* \textit{\Engma u{\vz} u\vt\vs u \vc o\vs u kar bole\vd} \\
% (Intended: `You hit me with that fish.')
% \a \ljudge* \textit{\Engma u{\vz} \vc o\vs u u\vt\vs u kar bole\vd} \\
% (Intended: `You hit me with that fish.')
% \xe

This does not otherwise change the syntactic behavior of the noun---if the argument in question is the object without the demonstrative postposition, it remains the syntactic object and verbal agreement still takes place.

\section{Ditransitives}

Geobo{\engma} is secundative, so for inherently ditransitive verbs like \textit{\'cneu} `to give', the syntactic object of the verb is the recipient. The theme can be included as a postpositional argument using the instrumental \textit{\vc o}.

% \ex
% \begingl
% Meo[mother]@
% -\l[\sc -poss]@
% -e{\vd}[\sc -1]
% geob[goblin]
% de{\vz}[cat]
% \vc o[\sc ins]@
% -\vs u[\sc -3.anim]
% \'cneu[give]@
% -meu[\sc -3.dom]
% \glft `The goblin gave my mother a cat.'
% \endgl
% \xe

Benefactive ditransitives can be derived from many verbs that are not inherently ditransitive by addition of the affix \textit{-tu-} before the agreement suffix but after the negation affix (if any).

% \ex
% \begingl
% Meo[mother]@
% -\l[\sc -poss]@
% -ar[-2]
% te{\vd}[1]
% \vs\vr eo{\vd}[eel]
% \vc o[\sc ins]@
% -\vs u[\sc -3.anim] 
% \vr u[cook]@
% -keo[-{\sc neg.dir}]@
% -tu[-{\sc ben}]@
% -g[-{\sc 3.gob}]
% \glft `I won't cook eel for your mother.'
% \endgl
% \xe

Since the theme is optional in these `ditransitives', this is also an effective way of describing an action in which the theme is unknown or unspecified, effectively serving as another argument-reduction strategy.

% \ex
% \begingl
% Meo[mother]@
% -\l[\sc -poss]@
% -eog[\sc -3.gob]
% \vS krik[Skreek]
% eog[\sc 3.gob]
% de\vn\vr e[sing]@
% -tu[\sc -ben]@
% -g[\sc -3.gob]
% \glft `Skreek sings for his mother.'
% \endgl
% \xe

\section{`Adjectives'}

Strictly speaking, there is not a separate syntactic class known as `adjectives' in Geobo{\engma}. What appear to be adjectives are actually verb forms---an uninflected verb can be used attributively by placing it directly before the noun it modifies.

% \pex
% \a
% \begingl
% \'Cre\'z[alive]
% \vt\'ne\vl[stick]
% \glft  `Flourishing branch'
% \endgl
% \a 
% \begingl
% \vT\'nel[stick]
% ban[sun]
% \'cre\'z[alive]@
% -im[\sc -3.inan]
% \glft `The sun made the branch flourish.'
% \endgl
% \xe

In more complex NPs containing possession or quantification, the location of the adjective can affect the meaning of a phrase.

% \pex
% \a
% \begingl
% \'Cre\'z[alive]
% \vt\'ne{\vl}[stick]
% da\'n[one]@
% -a\l[-{\sc poss}]@
% -m[-{\sc 3.inan}]
% \glft  `One of the flourishing branches'
% \endgl
% \a
% \begingl
% \vT\'ne{\vl}[stick]
% \'cre\'z[alive]
% da\'n[one]@
% -a\l[-{\sc poss}]@
% -m[-{\sc 3.inan}]
% \glft  `The flourishing one of the branches'
% \endgl
% \xe

As we've already seen in earlier examples, originally causative verbs can serve practically as predicative adjectives when inflected with a dummy subject

% \ex
% \begingl
% \Engma u\vz[fish]
% es[\sc 3.abst]
% nono\vn[enlarge]@
% -\vs u[\sc -3.anim]
% \glft `The fish is big.' (lit., `It enlarges the fish.')
% \endgl
% \xe

By default, bare verbs used attributively (i.e., as adjectives in the above examples) carry the meaning characteristic to the \textit{object} of that verb---e.g., \textit{\'cre\'z} means `alive'/`flourishing' not `live-giving', \textit{uh} means `dead' not `deadly', etc. In this respect, they could be called `passive participles.' Corresponding active participles, with meanings more characteristic of the subjects of these verbs, can be formed using the causative derivational prefix \textit{(o)c-} (e.g., \textit{oc\'cre\'z} `live-giving', \textit{cuh} `deadly', etc.). While this prefix technically also derives a causative verb, due to the nature of Geobo{\engma} verbs, the resulting verb is probably more often used as an attributive or predicative adjective than as a causative verb proper.

% \pex
% \a
% \begingl
% oz[\sc caus]@
% -zdeu[-bite]
% \engma u\vz[fish]
% \glft `That bitey fish'
% \endgl
% \a
% \begingl
% \engma u\vz[fish]
% es[\sc 3.abst]
% oz[\sc caus]@
% -zdeu[-bite]@
% -\vs u[\sc -3.anim]
% \glft `That fish tends to bite.' (lit., `It makes that fish bite.')
% \endgl
% \xe

Geobo{\engma} lacks separate comparative forms of the verb. In order to make predicative comparisons, the verb is turned into a benefactive ditranstive using the affix \textit{-tu-}. The `lesser' of the compared items then serves as the syntactic object of the verb, with the `greater' item as the optional instrumental argument, and the dummy pronoun remains the syntactic subject.

% \ex 
% \begingl
% Reu[house]@
% -\l[\sc -poss]@
% -e\vd[\sc -1]
% es[\sc 3.abst]
% \engma u\vz[fish] 
% \vc o[\sc ins]@
% -\vs u[\sc -3.anim]
% nonon[enlarge]@
% -tu[\sc -ben]@
% -m[\sc -3.inan]
% \glft `The fish is bigger than my house.' (lit., `It enlarges the fish for my house.')
% \endgl
% \xe

Note that the verb still agrees with the syntactic object, so unlike in the non-comparative example, the verb agrees with the noun that is less well-described by the adjective in question. This means that in such comparisons, the standard of comparison cannot be omitted.

\section{Focus \& Word Order}\label{focus}

To the casual observer, Geobo{\engma} seems superficially OSV. While it is true that this pattern is certainly the most commonly seen in the language, the reality is more complicated. This typical ordering is actually due to the focus marking that lies at the core of Geobo{\engma} sentence structure.

In any given sentence, whichever argument is the focus is moved to a sentence-initial position. 

% \ex
% \begingl
% A\l k[dog]
% \vS krik[Skreek]
% bo\vl\vs u[strike:{\sc 3.anim}]
% \glft `It was the dog that Skreek hit.'
% \endgl
% \xe

% \ex
% \begingl
% \vS krik[Skreek]
% a\l k[dog]
% bo\vl\vs u[strike:{\sc 3.anim}]
% \glft `It was Skreek who hit the dog.'
% \endgl
% \xe

Before this movement, the sentence is underlyingly SOV, so if a complement other than the subject or object is the focus, this is the order in which those arguments will appear.

% \ex
% \begingl
% \vT\vn e{\vl}[stick]
% \vc om[\sc ins:3.inan]
% \vS krik[Skreek]
% a\l k[dog]
% bo\vl\vs u[strike:{\sc 3.anim}]
% \glft `It was with a stick that Skreek hit the dog.'
% \endgl
% \xe

Only arguments can be fronted like this, so if the verb or an adjunct is the focus, the object is moved to this position in their stead. If one wishes to emphatically focus a non-argument, a preceding clause can be formed with the verb \textit{\'se} `to do, to make, to cause'

% \pex Q: `When did Skreek hit the dog?'
% \a \ljudge*
% \begingl
% Da\'s[yesterday]
% \vS krik[Skreek]
% a\l k[dog]
% bo\vl\vs u[strike:{\sc 3.anim}]
% \endgl
% \a 
% \begingl
% A\l k[dog]
% \vS krik[Skreek] 
% da\'s[yesterday]
% bo\vl\vs u[strike:{\sc 3.anim}]
% \glft A: `Skreek hit the dog yesterday.'
% \endgl
% \a 
% \begingl
% Es[\sc 3.abst]
% da\'s[yesterday]
% \'se[do]@
% s[\sc :3.abst]
% \nogloss{,}
% a\l k[dog]
% \vS krik[Skreek]
% bo\vl\vs u[strike:{\sc 3.anim}]
% \glft A: `It was \emph{yesterday} that Shreek hit the dog.'
% \endgl
% \xe

If the object is not in focus and would be represented with a pronoun, it can be omitted from the sentence. However, this is not the case with any other argument (i.e., subjects must always be explicitly included).

% \ex
% \begingl
% \glpreamble A: \textit{A\l k \vS krik da\'s bo\vl\vs u.} (`Skreek hit a dog yesterday.')\\B: \textit{\'Ne\l?} (`Really?')\endpreamble
% Ip[yes]
% \vt\vn e{\vl}[stick]
% \vc om[\sc ins:3.inan]
% eog[{\sc 3.gob}]
% bo\vl\vs u[strike:{\sc 3.animl}]
% \glft `Yes, it was with a stick that he hit (it).'
% \endgl
% \xe

\section{Relative Clauses}

Geobo{\engma} relative clauses are formed via adjoined clauses. The relativized noun must be fronted in both clauses, and therefore only arguments can be relativized. The complementizer \textit{deo\l} is placed clause-initially in the relativized clause, which is then directly followed by the main clause. The argument in question can be either restated in the main clause (often with a demonstrative) or replaced with a pronoun.

% \pex
% \a
% \begingl
% Deo{\l}[{\sc c}]
% oh[person]
% te{\vd}[1]
% \vz e\'zeog[see:{\sc 3.gob}]
% \nogloss{,}
% oh[person]
% \'no\vd eog[{\sc dist:3.gob}]
% reu[home]
% eumim[go.to:{\sc 3.inan}]
% \endgl
% \a
% \begingl
% Deo{\l}[{\sc c}]
% oh[person]
% te{\vd}[1]
% \vz e\'zeog[see:{\sc 3.gob}]
% eog[{\sc 3.gob}]
% reu[home]
% eumim[go.to:{\sc 3.inan}]
% \glft `The person who I saw went home.'
% \endgl
% \xe

The order of the relativized and main clause is determined by whether the relative clause is restrictive or not. In restrictive relative clauses, the relative clause precedes the main clause, while in non-restrictive relative clauses, the relative clause occurs afterwards.

% \ex
% \begingl
% Meo\l e\vd[mother:{\sc poss:1}]
% reu[home]
% eumim[go.to:{\sc 3.inan}]
% \nogloss{,}
% deo{\l}[{\sc c}]
% eog[{\sc 3.gob}]
% geob[goblin]
% \vz eog[\sc cop:3.gob]
% \glft `My mother, who is a goblin, went home.'
% \endgl
% \xe

Content clauses, in which the subordinate clause serves as an argument of the main sentence, are expressed using relative clauses modifying the dummy noun \textit{as}. Historically, \textit{as} could be roughly translated `case, situation, affair', but in the modern language it is not used outside of the formation of content clauses

% \ex
% \begingl
% Deo{\l}[{\sc c}] 
% geob[goblin]
% meo\l ar[mother:{\sc poss:2}]
% \vz eog[\sc cop:3.gob]
% \nogloss{,}
% as[{\sc c}]
% te{\vd}[1]
% \vz e\'zes[see:{\sc 3.abst}]
% \glft `I saw that your mother is a goblin.' (lit., `I saw the fact that your mother is a goblin.')
% \endgl
% \xe

Most other subordinating conjunctions in English are approximated by using the above strategy combined with various prepositions or verbs.

% \ex 
% \begingl
% Deo\l[\sc c]
% dizd[salt]
% \engma u\vz[fish]
% pidim[possess:{\sc 3.inan}]
% \nogloss{,}
% as[\sc c]
% euźdes[due\_to:{\sc 3.abst}]
% te\vd[\sc 1] 
% bo\vs kem[consume:{\sc neg.dir:3.inan}]
% \glft `I didn't eat the fish because it was salty.' (lit., `Due to the fact that the fish had salt, I didn't eat it.')
% \endgl
% \xe 

%\chapter{Semantics}

\section{Aspect Particles}

Geobo{\engma} is tenseless, only specifying tense if doing so is necessary in context and doing so using temporal adverbs, which are typically placed before the verb.

% \pex 
% \a 
% \begingl 
% bostu[eat:{\sc nmz}]
% te\vd[1]
% da\'s[yesterday]
% \'ses[do:{\sc 3.abst}]
% \glft `I ate yesterday.'
% \endgl
% \a 
% \begingl 
% bostu[eat:{\sc nmz}]
% te\vd[1]
% \vs a\vl\vn[someday]
% \'ses[do:{\sc 3.abstr}]
% \glft `I'll eat later.'
% \endgl 
% \xe

However, Geoboŋ does systematically mark aspect through use of particles, which occur at the beginning of the subject-less VP (that is, after the subject and before the object and other arguments should they not be fronted or otherwise moved).

%insert a syntax tree or smth here eventually

\subsubsection{Habitual--\textit{go}}

This particle indicates that something is always or virtually always the state of affairs. It is used in contexts where a gnomic or generic would be used in other languages:

% \ex
% \begingl
% Eu\engma r[sky]
% es[\sc 3.abstr]
% go[\sc hab]
% peoda\vn[blue:{\sc 3.ntfrc}]
% \glft `The sky is blue'
% \endgl
% \xe

as well as in contexts that describe habitual actions that still go on in the present and are expected to continue to go on for the foreseeable future:

% \ex 
% \begingl
% \Engma u\vz[fish]
% kar[\sc 2]
% go[\sc hab]
% \vr um[cook:{\sc 3.inan}]
% \nogloss{!}
% \glft `You \emph{always} cook fish!'
% \endgl
% \xe

Since aspect particles scope over negation, applying \textit{go} to a negated verb means `never'. If one wants to scope negation over the aspect, the whole clause must be embedded under a negative verb.

% \pex 
% \a 
% \begingl
% \Engma u\vz[fish]
% meo[mother]
% \vr ukem[cook:{\sc neg.dir:3.inan}]
% go[\sc gno]
% \glft `Mom never cooks fish.'
% \endgl
% \a 
% \begingl
% Deo\l[\sc c]
% \engma u\vz[fish]
% te\vd[\sc 1]
% \vr um[cook:{\sc 3.inan}]
% go[\sc gno]
% \nogloss{,}
% as[\sc c]
% es[\sc 3.abstr]
% his\engma keos[real:{\sc neg.dir:3.abstr}]
% \glft `It's not the case that I always cook fish.'
% \endgl
% \xe

\subsubsection{Completive---\textit{o\'sk}}

The completive particle \textit{o\'sk} is used when discussing an eventuality that has already occurred in relation to the time being discussed. It can often be translated with the English perfect or with the word `already'.

% \ex 
% \begingl
% Bostu[eat:{\sc nmz}]
% te\vd[\sc 1]
% o\'sk[\sc cmpl]
% \'ses[do:{\sc 3.abstr}]
% \glft `I've already eaten/drunk.'
% \endgl
% \xe

When used with a negative verb, it can be more accurately translated as `still haven't.'

% \ex 
% \begingl
% Bostu[eat:{\sc nmz}]
% te\vd[\sc 1]
% o\'sk[\sc cmpl]
% \'sekeos[do:{\sc neg.dir:3.abstr}]
% \glft `You still haven't eaten.'
% \endgl
% \xe

\subsubsection{Retrospective Habitual---\textit{pu}}

The retrospective habitual particle \textit{pu} is used to describe an eventuality that was once generally, habitually, or always the case, but has since ceased to be. It's usually best translated with `used to', `once', or `would X' in English.

% \ex 
% \begingl
% \Engma u\vz[fish]
% te\vd[\sc 1]
% pu[\sc rhab]
% \vr um[cook:{\sc 3.inan}]
% \glft `I used to cook fish' (but I've since stopped for some reason)
% \endgl
% \xe

This particle scopes over negation, so using it with a negative verb conveys that it used to be the case that they generally did not do a thing, but that now they do that thing (or at least have done so once). To merely negate the former habituality of an action, one must embed the whole clause under a negative verb, just as one must do to force negation to scope over the habitual particle.

% \pex 
% \a
% \begingl 
% \vS\vr eo\vd[snake]
% te\vd[\sc 1]
% pu[\sc rhab]
% bolkeo\vs u[strike:{\sc neg.dir:3.animl}]
% \glft `I used to never fight snakes.' (but now I've started doing it)
% \endgl 
% \a
% \begingl
% Deo\l[\sc c]
% \vs\vr eo\vd[snake]
% te\vd[\sc 1]
% pu[\sc rhab]
% bo\vl\vs u[strike:{\sc 3.animl}]
% \nogloss{,}
% as[\sc c]
% es[\sc 3.abstr]
% his\engma keos[real:{\sc neg.dir:3.abstr}]
% \glft `I didn't used to fight snakes.' (I never stopped fighting snakes!)
% \endgl 
% \xe

\subsubsection{Change-of-state---}

The change-of-state particle %\textit{\vl i}%
 indicates that an eventuality is new or novel in some way, a change to the previous status-quo or otherwise contrastive to the prior context.

% \ex
% \begingl
% \vS\vr eo\vd[snake]
% \'zreo\vn\vz eog[scare:{\sc 3.gob}]
% go[\sc gno]
% eo[and]
% eog[\sc 3.gob]
% \vl i[\sc cos]
% zdeu\vs u[bite:{\sc 3.animl}]
% \glft `Snakes scare him, but he bit (one)!'
% \endgl
% \xe

It's often contrasted with a status quo or former status quo expressed with the gnomic or retrospective habitual particles. However, there is no requirement that they be paired together; the change-of-state particle can be used on its own to merely imply the departure from the status quo without making the status quo itself explicit.

% \ex
% \begingl
% \Engma u\vz[fish]
% te\vd[\sc 1]
% \vl i[\sc cos]
% bo\vs im[eat:{\sc 3.inan}]
% \glft `I eat fish now' (we both know I used to be a vegetarian)
% \endgl
% \xe

In these contexts, it's often translated with `now' or `anymore'. 

In most dialects, the aspect particles are mutually exclusive. However, in some dialects 
%\textit{\vl i} 
can be combined with other aspectual particles, in which case it directly precedes them.

% \ex \ljudge{?}
% \begingl
% \Engma u\vz[fish]
% \vS krik[Skreek]
% pu[\sc rhab]
% bo\vs kem[eat:{\sc neg.dir:3.inan}]
% eo[and]
% \vs\vr eo\vd[eel]
% eog[\sc 3.gob]
% \vl i[\sc cos]
% go[\sc hab]
% bo\vs im[eat:{\sc 3.inan}]
% \glft `Skreek never used to eat fish but he's always eaten eels.'
% \endgl
% \xe

However, this usage of 
%\textit{\vl i} 
is markedly dialectical and not more generally acceptable.

\section{Modality}

Modal affixes are placed directly following the verb stem, prior to negation (if present).

\section{Discourse Markers}

%%%%%%%%%%%


\subsubsection{Possibility---}

The possibility particle 
%\textit{\vz i}% 
indicates that an event is possible. It generally only represents circumstantial or epistemic possibility, though it can take on a deontic flavor if combined with 
%\textit{\vr o} 
(see \ref{sec:ro}). No obvious distinction is made between epistemic and circumstantial possibility in Geobo{\engma}, so which the speaker intends must be decided based on context alone.

% \pex 
% \a
% \begingl
% \glpreamble
% Q: What can Skreek cook?
% \endpreamble
% \nogloss{\normalfont A:}
% \Engma u\vz[fish]
% eog[\sc 3.gob]
% \vr um[cook:{\sc 3.inan}]
% \vz i[\sc psbl]
% \glft \phantom{A: }`He is able to cook fish.'
% \endgl
% \a 
% \begingl
% \glpreamble
% Q: What do you think Skreek is doing right now?
% \endpreamble
% \nogloss{\normalfont A:}
% \Engma u\vz[fish]
% eog[\sc 3.gob]
% \vr um[cook:{\sc 3.inan}]
% \vz i[\sc psbl]
% \glft \phantom{A: }`He could be cooking fish.'
% \endgl
% \xe

\section{Discourse Particles}

In addition to the particles indicating TAM, Geobo{\engma} possess a series of more discourse-focused particles which indicate the attitude of the speaker towards the propositional content of their utterance---they tend to communicate the illocutionary rather than locutionary act. They always occur sentence-finally, following even the TAM particles if any are present.

\subsubsection{Question---\textit{\'ne\l}}

The question particle \textit{\'ne\l} is appended to the end of a sentence when the sentence is a yes/no question. It's also used to ask what would be considered a tag question in English, wherein the speaker believes what they're saying is true and is seeking agreement. Intonation often plays a role.

\ex
\begingl
Meo\l ar[mother:{\sc poss:2}]
reu[home]
eumim[go\_to:{\sc 3.inan}]
o\'sk[\sc cmpl]
\'ne\l[\sc q]
\glft `Did your mother already go home?' or `Your mother already went home, right?'
\endgl 
\xe

Note that \textit{\'ne\l} is only used if the speaker is seeking information or agreement---while yes/no questions are often used for other illocutionary acts in other languages, these are indicated by different particles in Geobo{\engma}. The above sentence would not be used, for instance, if the question was actually intended as a request for the listener to send their mother home.

\textit{\'Nel} can also be used for embedded questions, serving as an equivalent to the English `whether'

% \ex
% \begingl
% Deo\l[\sc c]
% meo\l ar[mother:{\sc poss:2}]
% reu[home]
% eumim[go\_to:{\sc 3.inan}]
% o\'sk[\sc cmpl]
% \'ne\l[\sc q]
% \nogloss{,}
% as[\sc c]
% te\vd[\sc 1]
% \vt\vl irckeos[know:{\sc neg.dir:3.abstr}]
% \glft `I don't know whether your mother already went home.'
% \endgl
% \xe

Emphatic questions can be unambiguously asked by turning the entire clause, including \textit{\'ne\l}, into an embedded question under a verb directly questioning the truth of the utterance, essentially doubling-up on the questioning nature of the utterance.

\ex
\begingl
Deo\l[\sc c]
meo\l ar[mother:{\sc poss:2}]
reu[home]
eumim[go\_to:{\sc 3.inan}]
o\'sk[\sc cmpl]
\'ne\l[\sc q]
\nogloss{,}
as[\sc c]
es[\sc es]
hisnes[real:{\sc 3.abstr}]
\'ne\l[\sc q]
\glft `Did your mother already go home?' (lit., `Is it true whether your mother already went home?')
\endgl 
\xe

This is generally seen as quite forceful and is thus impolite outside of very casual or very formal situations.

\textit{\'Ne\l} is also often used on its own as a dialogue filler to ellicit agreement or encourage elaboration, much like the English `right?', `uh-huh?', or `really?'

\subsubsection{Request---}\label{sec:ro}

The request particle is appended to the end of a sentence when the speaker wants to turn the sentence into a request or suggestion for action. Often this manifests as it being directly appended to the end of a clause describing the content of the request:

% \ex
% \begingl
% Disd[salt]
% \vc om[\sc ins:3.inan]
% kar[\sc 2]
% \'cneu\vd[give:{\sc 1}]
% \vr o[\sc req]
% \glft `Would you please hand me the salt?' or `How about you hand me the salt?'
% \endgl
% \xe

However, it can also be used with shorter utterances, even appended simply to the end of a noun if the context is right.

% \ex
% \begingl
% Disd[salt]
% \vr o[\sc req]
% \glft `Salt, please?' or `How about the salt?'
% \endgl
% \xe

It can be even further abstracted, used with sentences that do not directly reference the requested eventuality whatsoever and only imply what the request itself actually is.

% \ex
% \begingl
% Meo\l ar[mother:{\sc poss:2}]
% reu[home]
% eumim[go\_to:{\sc 3.inan}]
% \vr o[\sc req]
% \glft `Your mother went home, hint-hint.'
% \endgl
% \xe

Like all discourse particles, %\textit{\vr o}
 scopes over negation. In order to negate a request rather than turning the sentence into a request for the negative, one must negate the verb 
 %\textit{\vr o\'s} and embed the request under it.

% \ex
% \begingl
% Deo\l[\sc c]
% disd[salt]
% \vc om[\sc ins:3.inan]
% kar[\sc 2]
% \'cneu\vd[give:{\sc 1}]
% \vr o[\sc req]
% \nogloss{,}
% as[\sc c]
% \vc os[\sc ins:3.abstr]
% te\vd[\sc 1]
% \vr o\'sog\'zar[request:{\sc neg.inv:2}]
% \glft `I'm not asking you to pass me the salt.'
% \endgl
% \xe

\subsubsection{Command---}

The command particle 
%\textit{\vd et} 
is appended to the end of a sentence to indicate that the speaker is commanding the hearer to do something. It is essentially a more forceful version of 
%\textit{\vr o}. 

% \ex
% \begingl
% Disd[salt]
% \vc om[\sc ins:3.inan]
% kar[\sc 2]
% \'cneu\vd[give:{\sc 1}]
% \vd et[\sc imp]
% \glft `Hand me the salt!'
% \endgl
% \xe

% Unlike \textit{\vr o}, however, \textit{\vd et} cannot be used for negative commands/prohibitions.

% \pex 
% \a 
% \begingl
% Disd[salt]
% \vc om[\sc ins:3.inan]
% kar[\sc 2]
% \'cneukeo\vd[give:{\sc neg.dir:1}]
% \vr o[\sc req]
% \glft `Please don't hand me the salt.'
% \endgl
% \a \ljudge{\#}
% \begingl
% Disd[salt]
% \vc om[\sc ins:3.inan]
% kar[\sc 2]
% \'cneukeo\vd[give:{\sc neg.dir:1}]
% \vd et[\sc imp]
% \glft Intended: `Don't hand me the salt!'
% \endgl
% \xe

% Just like \textit{\vr o}, \textit{\vd et} can be appended to a mere noun as an order for that noun and can be appended to more opaque utterances depending on context. 

% \ex 
% \begingl
% Bostu[food]
% de\vz[cat]
% \dbl eo\lam okeom[have:{\sc neg.dir:3.inan}]
% \vd et[\sc imp]
% \glft `The cat doesn't have food (Feed the cat!)'
% \endgl
% \xe

Negation can scope over the command particle using the verb 
%\textit{\vd et\'s} 
`to command'. However, since simply using the command particle with a negative verb doesn't necessarily lead to a prohibitive meaning as one would expect were the imperative to scope over the negation, many speakers use that construction and treat the negation as scoping over the imperative.

% \pex
% \a 
% \begingl
% Deo\l[\sc c]
% de\vz[cat]
% kar[\sc 2]
% arc\vs u[feed:{\sc 3.animl}]
% \vd et[\sc imp]
% \nogloss{,}
% as[\sc c]
% \vc os[\sc ins:3.abstr]
% es[\sc 3.abstr]
% \vd et\'sog\'zar[command:{\sc neg.inv:2}]
% \endgl
% \a \ljudge{?}
% \begingl
% De\vz[cat]
% kar[\sc 2]
% arckeo\vs u[feed:{\sc neg.dir:3.animl}]
% \vd et[\sc imp]
% \glft `You don't have to feed the cat.'
% \endgl
% \xe 

The latter construction is seen as childish and `incorrect', but the former is seen as somewhat stilted and formal, so which is chosen largely depends on the context.

\subsubsection{Prohibitive---\textit{iz}}

The prohibitive particle \textit{iz} is appended to the end of a sentence to indicate that the speaker forbids the hearer from doing something. Usually, it'll be something referred to directly by the sentence the particle is attached to, though it doesn't necessarily have to directly be the proposition expressed by the rest of the clause.

% \ex 
% \begingl
% \Engma u\vz[fish]
% kar[\sc 2]
% bo\vs im[eat:{\sc 3.inan}]
% iz[\sc proh]
% \glft `Don't eat the fish!'
% \endgl
% \xe 

Like the other deontic particles, it can be used with a single noun, indicating that the hearer should avoid or not interact with that particular noun (in whatever way is context appropriate). It can also be used more obliquely, prohibiting something more indirectly related to the clause it's attached to

% \ex
% \begingl
% \Engma u\vz[fish]
% te\vd[\sc 1]
% bo\vs im[eat:{\sc 3.inan}]
% iz[\sc proh]
% \glft `Don't make me eat fish!'
% \endgl
% \xe

Related constructions formed with \textit{iz} can often approximate the meanings of other discourse particles, though usually with differences in affect or formality. Like other discourse particles, \textit{iz} scopes over negation, and therefore using it with a negative verb can be interpreted as an imperative. 

% \ex 
% \begingl
% \Engma u\vz[fish]
% kar[\sc 2]
% bo\vs kem[eat:{\sc neg.dir:3.inan}]
% iz[\sc proh]
% \glft `You've got to eat the fish!' (lit., `You mustn't not eat the fish')
% \endgl
% \xe

However, this construction can be seen as awkward in many contexts, as it is often blocked by %\textit{\vd et},
 so it's used sparingly if ever by many speakers.

Negation can scope over the prohibitive using the verb \textit{i\l\'s} `to forbid', which behaves essentially like a permissive.

% \ex 
% \begingl
% Deo\l[\sc c]
% \engma u\vz[fish]
% kar[\sc 2]
% bo\vs im[eat:{\sc 3.inan}]
% iz[\sc proh]
% \nogloss{,}
% as[\sc c]
% \vc os[\sc ins:3.abstr]
% es[\sc 3.abstr]
% i\l\'sog\'zar[forbid:{\sc neg.inv:2}]
% \glft `You're not forbidden from eating fish.'
% \endgl
% \xe

However, unlike the permissive particle %\textit{eu\vs},
 this construction is highly distancing and heavily implicates that the speaker isn't the relevant one when it comes to deciding whether the thing in question is permitted or forbidden.

\textit{Iz} is also used to mean `no' when the speaker is saying it in a forward-thinking context in which they want to prevent some future eventuality from happening via their `no'.

\subsubsection{Permissive---} 

The permissive particle %\textit{eu\vs} 
is appended to the end of a sentence to indicate that the speaker is allowing the hearer to do something (typically but not always what the sentence overtly describes) but doesn't have any strong feelings one way or the other, positive or negative. It's the Geobo{\engma} equivalent of shrugging and saying `Sure, I don't really give a shit,' but is also more widely used to indicate that the hearer may do something but isn't necessarily being encouraged to do that thing.

% \ex
% \begingl
% \vZ leuoh[child]
% \vd eo\'s[tree]
% pogebim[climb:{\sc 3.inan}]
% eu\vs[\sc perm]
% \glft `Children (you) may climb trees.'
% \endgl
% \xe

Like many of the other particles, it can be used with a simple noun in many contexts.

% \ex 
% \begingl
% \glpreamble
% Q: Would you like something to eat or drink?
% \endpreamble
% \nogloss{\normalfont A:}
% Teu\l b[still\_water]
% eu\vs[\sc perm]
% \glft\phantom{A: }`Water's fine.'
% \endgl
% \xe

Like related particles, it can also be used with more opaque sentences if context makes it clear what's being permitted. 

Scoping negation over the permissive particle is done via embedding with the verb 
%\textit{eu\dbl\'s} `to permit.'

% \ex 
% \begingl
% Deo\l[\sc c]
% \vz leuoh[child]
% \vd eo\'s[tree]
% pogebim[climb:{\sc 3.inan}]
% eu\vs[\sc perm]
% \nogloss{,}
% as[\sc c]
% \vc os[\sc ins:3.abstr]
% es[\sc 3.abstr]
% eu\dbl\'sog\'zar[permit:{\sc neg.inv:2}]
% \glft `Children (you) are not allowed to climb trees.'
% \endgl
% \xe

While this may seem at first glance to be identical to the use of the prohibitive particle, it doesn't constitute a direction illocutionary prohibition the way \textit{iz} does. Instead, it comes off as a statement about the way things are and thus strongly implicates that the speaker is not the one in control of whether the hearer is permitted or prohibited from doing whatever is described.

%\textit{Eu\vs} 
is also used as an equivalent for `yes' when the speaker wants to express lethargy or apathy towards their agreement.

\subsubsection{Positive Attitude---\textit{ip}}

The positive attitude particle \textit{ip} is appended to the end of a sentence to indicate that the speaker has a positive outlook on the eventuality described by the sentence in question---that they believe that eventuality is/was good, fortunate, natural, and/or just.

% \ex
% \begingl
% \vS krik[Skreek]
% es[\sc 3.abstr]
% uheog[kill:{\sc 3.gob}]
% ip[:-{)}]
% \glft `Yay, Skreek died!'
% \endgl
% \xe

\textit{Ip} is also used as an equivalent for `yes' when the speaker wants to express enthusiasm or an otherwise positive attitude toward their agreement.

\subsubsection{Negative Attitude---\textit{hu\l}}

The negative attitude particle \textit{hu\l} is appended to the end of a sentence to indicate that the speaker has a negative outlook on the eventuality described by the sentence in question---that they believe that the eventuality is/was bad, unfortunate, unjust, unnatural, or otherwise unnacceptable.

% \ex 
% \begingl
% \vS krik[Skreek]
% es[\sc 3.abstr]
% uheog[kill:{\sc 3.gob}]
% hu\l[:-{(}]
% \glft `Oh no, Skreek died!'
% \endgl
% \xe

\textit{Hu\l} is also used as an equivalent for `yes' when the speaker wants to express a negative attitude toward their agreement.

%\subsubsection{Desiderative---\textit{oc}}



\part{Sample Texts \& Translations}



As one of the few relics of the animacy hierarchy, there are different morphemes for verbal negation depending on the noun classes of the subject and object. If the subject is equal to or higher on the animacy hierarchy than the object, the affix \textit{-(o)keo-} is placed directly after the root. If the subject is lower on the animacy hierarchy than the object, the affix \textit{-og\'z(u)-} is used instead. 

% \pex
% \a
% \begingl
% Geob[goblin] 
% te\vd[1]
% bol[strike]@
% -keo[-{\sc neg.dir}]@
% -g[-{\sc 3.gob}]
% \glft `I didn't hit a goblin.'
% \endgl
% \a
% \begingl
% Geob[goblin] 
% te\vd[1]
% bol[strike]@
% -og\'z[-{\sc neg.inv}]@
% -e\vd[-{\sc 1}]
% \glft `A goblin didn't hit me.'
% \endgl
% \xe
