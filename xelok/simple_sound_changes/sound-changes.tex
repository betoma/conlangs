\documentclass[a4paper,11pt,article,oneside]{memoir}
\usepackage{fontspec}
\usepackage{amsmath} % a pretty standard package to enlarge your inventory of symbols
\usepackage{amssymb} % another common package for symbols
\usepackage[hidelinks]{hyperref} % enables hyperlinks in your document (no worries -- they show up only on the screen. When you print a hard copy, the colored boxes aren't there)
\usepackage{url} % helps typeset URLs properly, typically with the command \url
\usepackage[margin=.75in]{geometry} % page layout
\usepackage{booktabs} % creates beautiful and professional tables
\usepackage{multirow}
\usepackage{textcomp}
\usepackage{expex}
\usepackage{tablefootnote}
\usepackage[calc,english]{datetime2}
\usepackage{bookmark}

\setmainfont{Charis SIL}[CharacterVariant=43:1]

\setlength{\parindent}{0em}
\setlength{\parskip}{1ex}
%\linespread{1.1}

%---generic symbols---%
\newcommand{\dao}{$\to$}
\newcommand{\nm}{\symbol{"2205}}
\newcommand{\ra}{\textgreater}
\newcommand{\ipkt}{·}
\newcommand{\til}{\textasciitilde}

\newcommand{\ortho}[1]{$\langle$#1$\rangle$}
\newcommand{\bripa}[1]{[#1]}
\newcommand{\phipa}[1]{/#1/}

%---LANG'S ORTHO---%
%\^{} for lil hat
%\v{} for the hat that isn't dumb
%\u{u} for that one letter u
%\~{} for tilde
%\l for bar l
\newcommand{\lam}{λ}
\newcommand{\lambar}{ƛ}
\newcommand{\Engma}{Ŋ}
\newcommand{\espq}[1]{\textit{`#1'}}

%---IPA---%
%consonants
\newcommand{\bilaf}{ɸ}
\newcommand{\bilav}{β}
\newcommand{\labrox}{ʋ}
\newcommand{\latfric}{ɬ}
\newcommand{\latfrivoic}{ɮ}
\newcommand{\alvr}{ɹ}
\newcommand{\alvrap}{ɾ}
\newcommand{\alvlap}{ɺ}
\newcommand{\esh}{ʃ}
\newcommand{\ezh}{ʒ}
\newcommand{\paljstop}{ɟ}
\newcommand{\paljfric}{ʝ}
\newcommand{\egna}{ɲ}
\newcommand{\rette}{ʈ}
\newcommand{\retde}{ɖ}
\newcommand{\retesh}{ʂ}
\newcommand{\retezh}{ʐ}
\newcommand{\retna}{ɳ}
\newcommand{\egh}{ɣ}
\newcommand{\engma}{ŋ}
\newcommand{\vell}{ʟ}
\newcommand{\velr}{ʁ}
\newcommand{\glotstop}{ʔ}
\newcommand{\voih}{ɦ}
%vowels
\newcommand{\frno}{ø}
\newcommand{\bari}{ɨ}
\newcommand{\unru}{ɯ}
\newcommand{\unro}{ɤ}
\newcommand{\eps}{ɛ}
\newcommand{\oeps}{œ}
\newcommand{\sche}{ɘ}
\newcommand{\schwa}{ə}
\newcommand{\centruh}{ɜ}
\newcommand{\opno}{ɔ}
\newcommand{\aesh}{æ}
\newcommand{\oesh}{ɶ}
\newcommand{\centra}{ɐ}
\newcommand{\ahoh}{ɒ}
%diacritics and modifiers
\newcommand{\asp}{ʰ}
\newcommand{\lab}{ʷ}
\newcommand{\pal}{ʲ}
\newcommand{\jekt}{ʼ}
\newcommand{\rhot}{˞}
\newcommand{\sylb}{̩}
\newcommand{\lgth}{ː}
\newcommand{\vless}{̥}
\newcommand{\upvless}{̊}
\newcommand{\voicd}{̬}
\newcommand{\retr}{̠}
\newcommand{\lamino}{̻}
\newcommand{\apico}{̺}
\newcommand{\lowap}{̞}
\newcommand{\prstr}{ˈ}
\newcommand{\scstr}{ˌ}
\newcommand{\tiebar}{͡} 
\newcommand{\downstep}{ꜜ}

\lingset{glstyle=nlevel,numoffset=3em,textoffset=1.5em,exskip=.75ex,belowglpreambleskip=.25ex,aboveglftskip=.25ex}

\DTMnewdatestyle{eurodate}{%
    \renewcommand{\DTMdisplaydate}[4]{%
        \number##3.\nobreakspace%           day
        \DTMmonthname{##2}\nobreakspace%    month
        \number##1%                         year
    }%
    \renewcommand{\DTMDisplaydate}{\DTMdisplaydate}%
}

\DTMsetdatestyle{eurodate}

\title{X\~{e}\lam\~{o}k Sound Changes:\\{\Large A Cheat Sheet}}
\author{Bethany E. Toma}
\date{\today}

\counterwithout{section}{chapter}
\maxsecnumdepth{subsection}

\begin{document}

\maketitle

While X\~{e}\lam\~{o}k is distinct enough from its Esperanto ancestor to be reasonably considered a separate language at this juncture, its orthography does not reflect this. While syntactic and semantic changes are reflected to a fairly reasonable extent in the orthography, phonological changes are not accounted for. As a reult, to reasonably approximate the pronunciation of a X\~{e}\lam\~{o}k word from how it is written, one must understand which sound changes have occurred from the `original' (hereafter `Zamenhofian') Esperanto. 

Note that some sound changes listed apply only to the Lowlands dialect but not to the Highlands dialect, or vice-versa. This is indicated when it is the case.

\section*{Zamenhofian Esperanto Orthography}

The Esperanto pronunciation of a word tends to be roughly equivalent to its spelling in IPA, with the following exceptions:
\begin{center}
\begin{tabular}{ccccccc}
\ortho{c} & \ortho{\^{c}} & \ortho{\^{g}} & \ortho{\^{j}} & \ortho{\^{s}} & \ortho{\u{u}} & \ortho{v}\\[0.2cm]
\phipa{t\tiebar s} & \phipa{t\tiebar\esh} & \phipa{d\tiebar\ezh} & \phipa{\ezh} & \phipa{\esh} & \phipa{w} & \phipa{\labrox}
\end{tabular}
\end{center}

Primary stress falls on the penultimate syllable, and secondary stress \emph{tends} to fall on alternating syllables preceding it.

\section{Zamenhofian Esperanto to \textit{Esperantelo}}

\subsection{\phipa{\labrox} becomes \bripa{w} when it would naturally fit in the sonority hierarchy of a consonant cluster and \bripa{v} elsewhere}

\begin{center}
    \begin{tabular}{l}
        \labrox{} $\to$ w / C\_\\
        \labrox{} $\to$ v / \emph{elsewhere}    
    \end{tabular}
\end{center}
e.g., \espq{kvarto} \bripa{\prstr k\labrox ar.to} $\to$ \bripa{\prstr kwar.to}, \espq{evakui} \bripa{\scstr e.\labrox a\prstr ku.i} $\to$ \bripa{\scstr e.va\prstr ku.i}

\subsection{Stress moves to the last syllable of correlatives and forms of \espq{esti}}

e.g., \espq{tiel} \phipa{\prstr ti.el} $\to$ \phipa{ti\prstr el}, \espq{estas} \phipa{\prstr e.stas} $\to$ \phipa{e\prstr stas}, etc.

\subsection{\phipa{i} and \phipa{u} $\to$ \phipa{j} and \phipa{w} before a stressed vowel}

e.g., \espq{duo.no} \bripa{du\prstr ono} $\to$ \bripa{\prstr dwo.no}, \espq{tiel} \bripa{ti\prstr el} $\to$ \bripa{tjel}, etc.

\subsection{Nasals assimilate in place of articulation to a following obstruent}
\begin{center}
    \begin{tabular}{l}
        N $\to$ m / \_\{p, b, f\}\\[0.1cm]
        N $\to$ n / \_\{t, d, t\tiebar\esh, d\tiebar\ezh, s, z, \esh, \ezh\}\\[0.1cm]
        N $\to$ \engma / \_\{k, g\}
    \end{tabular}
\end{center}
e.g., \espq{enblovi} \bripa{en\prstr blowi} $\to$ \bripa{em\prstr blo.wi}

\subsection{Obstruents assimilate in voicing to a following obstruent}

e.g., \espq{absolute} \bripa{ab.so\prstr lu.te} $\to$ \bripa{ap.so\prstr lu.te}

\subsection{Sonorants assimilate in voicing to a preceding stop}

e.g., \espq{plua} \bripa{\prstr plu.a} $\to$ \bripa{\prstr pl\vless u.a}

\subsection{Clusters of mixed sibilants assimilate to the last sibilant}

e.g., \espq{dis\^{j}eti} \bripa{dis\prstr\ezh e.ti} $\to$ \bripa{di\ezh\prstr\ezh e.ti}

\subsection{Voiwel hiatuses are broken up by epenthetic consonants}

\begin{itemize}
    \item If both consonants are the same, the epenthetic consonant is a glottal stop.
    \begin{center}
    \begin{tabular}{lllll}
        \bripa{a.a} $\to$ \bripa{a\glotstop a} &
        \bripa{e.e} $\to$ \bripa{e\glotstop e} &
        \bripa{i.i} $\to$ \bripa{i\glotstop i} &
        \bripa{o.o} $\to$ \bripa{o\glotstop o} &
        \bripa{u.u} $\to$ \bripa{u\glotstop u}
    \end{tabular}
    \end{center}
    \item If the first non-low vowel is front, the epenthetic consonant is \bripa{j}
    \begin{center}
    \begin{tabular}{llll}
        \bripa{i.e} $\to$ \bripa{ije} &
        \bripa{i.a} $\to$ \bripa{ija} &
        \bripa{i.o} $\to$ \bripa{ijo} &
        \bripa{i.u} $\to$ \bripa{iju}\\
        \bripa{e.i} $\to$ \bripa{eji} &
        \bripa{e.a} $\to$ \bripa{eja} &
        \bripa{e.o} $\to$ \bripa{ejo} &
        \bripa{e.u} $\to$ \bripa{eju}\\
        \bripa{a.i} $\to$ \bripa{aji} &
        \bripa{a.e} $\to$ \bripa{aje}
    \end{tabular}
    \end{center}
    \item If the first non-low vowel is back, the epenthetic consonant is \bripa{w}
    \begin{center}
    \begin{tabular}{llll}
        \bripa{o.i} $\to$ \bripa{owi} &
        \bripa{o.e} $\to$ \bripa{owe} &
        \bripa{o.a} $\to$ \bripa{owa} &
        \bripa{o.u} $\to$ \bripa{owu}\\
        \bripa{u.i} $\to$ \bripa{uwi} &
        \bripa{u.e} $\to$ \bripa{uwe} &
        \bripa{u.a} $\to$ \bripa{uwa} &
        \bripa{u.o} $\to$ \bripa{uwo}\\
        \bripa{a.o} $\to$ \bripa{awo} &
        \bripa{a.u} $\to$ \bripa{awu}
    \end{tabular}
    \end{center}
\end{itemize}

\subsection{Velar obstruents become palatal before front vowels}

\begin{center}
k g $\to$ c \paljstop{} / \_\{i,e\}
\end{center}

e.g., \espq{kilogramo} \bripa{\scstr ki.lo\prstr gra.mo} $\to$ \bripa{\scstr ci.lo\prstr gra.mo}

\subsection{Diphthongs turn into stressed (long) monophthongs}

\begin{center}
\begin{tabular}{lllll}
    ij $\to$ i\lgth &
    uj $\to$ y\lgth &
    aj ej $\to$ e\lgth &
    oj ew $\to$ ø\lgth &
    aw $\to$ o\lgth
\end{tabular}%
\end{center}

Note that this only occurs when the glide is clearly part of a diphthong, not in contexts like prevocalically where it's more of a consonant.\\
e.g, \espq{kajto} \bripa{\prstr kaj.to} $\to$ \bripa{\prstr ke\lgth.to} but \espq{kajako} \bripa{ka\prstr ja.ko} $\to$ \bripa{ka\prstr ja.ko}

\subsection{Glides fricate intervocalically or word-initially (i.e., when it would be the sole onset of a syllable according to the maximum onset principle)}

\begin{center}
j w $\to$ \paljfric{} v / \#\_ , V\_V
\end{center}

e.g., \espq{ejakuli} \bripa{\scstr e.ja\prstr ku.li} $\to$ \bripa{\scstr e.\paljfric a\prstr ku.li} but \espq{ajna} \bripa{\prstr aj.na} $\to$ \bripa{\prstr e\lgth na} but \espq{anta\u{u}a} \bripa{an\prstr ta.wa} $\to$ \bripa{an\prstr ta.va}

\subsection{Glottal stop and glottal fricative merge (differs by dialect)}

\begin{center}
\textbf{\sc Lowlands:} \phipa{\glotstop} $\to$ \phipa{h} \\
\textbf{\sc Highlands:} \phipa{h} $\to$ \phipa{\glotstop}
\end{center}

\subsection{{\sc Lowlands:} Standalone obstruents are voiced intervocalically} 

e.g., \espq{\^{s}ipo} \bripa{\prstr\esh i.po} $\to$ \bripa{\prstr\esh i.bo} 

\subsection{Intervocalic geminates become single occurrences of the consonant in question}

e.g., \espq{dis\^{s}uti} \bripa{di\esh\prstr\esh u.di} $\to$ \bripa{di\prstr\esh u.di}

\subsection{Ablaut}

If the last vowel in a word is back rounded, the preceding vowel is rounded (regardless of intervening consonants).\\
e.g., \espq{kato} \bripa{\prstr ka.to} $\to$ \bripa{\prstr k\ahoh.to}, \espq{iros} \bripa{\prstr i.ros} $\to$ \bripa{\prstr y.ros}, \espq{ekzemplo} \bripa{ek\prstr sem.plo} $\to$ \bripa{ek\prstr søm.plo}

If the last vowel in a word is front, the preceding vowel is fronted (regardless of intervening consonants).\\
e.g., \espq{havis} \bripa{\prstr ha.vis} $\to$ \bripa{\prstr h\aesh.vis}, \espq{ofte} \bripa{\prstr of.te} $\to$ \bripa{\prstr øf.te}, \espq{seksumi} \bripa{sek\prstr su.mi} $\to$ \bripa{sek\prstr sy.mi}

\subsection{Vowels are nasalized before nasal consonants, and nasal consonants are deleted when they precede obstruents}

e.g., \espq{anka\u{u}} \bripa{\prstr\ahoh n.ko\lgth} $\to$ \bripa{\prstr\~{\ahoh}.ko\lgth}

\subsection{{\sc Highlands:} Standalone obstruents are voiced intervocalically}

e.g., \espq{\^{s}ipo} \bripa{\prstr\esh y.po} $\to$ \bripa{\prstr\esh y.bo}, \espq{dis\^{s}uti} \bripa{di\prstr\esh y.ti} $\to$ \bripa{di\prstr\ezh y.di}, \espq{anka\u{u}} \bripa{\prstr\~{\ahoh}.ko\lgth} $\to$ \bripa{\prstr\~{\ahoh}.go\lgth}

\subsection{Unstressed short vowels become schwa}

e.g., \espq{Esperanto} \bripa{\scstr e.spe\prstr r\~{\ahoh}.to} $\to$ \bripa{\scstr e.sp\schwa\prstr r\~{\ahoh}.t\schwa}, \espq{tajfuno} \bripa{te\lgth\prstr fu.no} $\to$ \bripa{te\lgth\prstr fu.n\schwa}

\section{\textit{Esperantelo} to X\~{e}\lam\~{o}k}

\subsection{\phipa{l} becomes \bripa{w} when it follows a vowel and does not precede a vowel}

\begin{center}
l $\to$ w / V\_C , V\_\#
\end{center}

\subsection{When \phipa{r} follows a vowel and does not preced a vowel, it rhotacizes the preceding vowel and disappears}

\begin{center}
    Vr $\to$ V\rhot{} / \_C , \_\#
\end{center}

\subsection{Schwa is deleted when the result wouldn't be a total mess sonority-wise}

e.g., \espq{Esperanto} \bripa{\scstr e.sp\schwa\prstr r\~{\ahoh}.t\schwa} $\to$ \bripa{e\prstr spr\~{\ahoh}t}

If the schwa is \emph{nasal} and \emph{precedes and obstruent}, that obstruent becomes nasal if voiced and voiced if voiceless. If this happens, the next vowel in the word is nasalized.

e.g., \espq{Esperantelo} \bripa{\schwa\scstr spe.r\~{\schwa}\prstr tø.l\schwa} $\to$ \bripa{sper\prstr d\~{ø}l}

If the schwa is \emph{rhotic} and \emph{precedes a coronal}, said coronal becomes retroflex.
\begin{center}
    \begin{tabular}{llllll}
        t $\to$ \rette &
        d $\to$ \retde &
        s $\to$ \retesh &
        z $\to$ \retezh &
        t\tiebar s t\tiebar\esh{} $\to$ t\tiebar\retesh &
        d\tiebar z d\tiebar\ezh{} $\to$ d\tiebar\retezh
    \end{tabular}
\end{center}
e.g., \espq{sardino} \bripa{s\schwa\rhot\,\prstr d\~{y}.n\schwa} $\to$ \bripa{s\retde\,\~{y}n}

\subsection{Liquids become glides after sonorants}

\begin{center}
    l r $\to$ j w / S\_
\end{center}

e.g., \espq{malebla} \bripa{\prstr mlebl} $\to$ \bripa{mjebl}

\subsection{Front vowels are raised and back vowels fronted after palatal consonants}

\begin{center}
    \begin{tabular}{@{}rrrl@{}}
        ø u $\to$ y & o $\to$ ø & \ahoh $\to$ \oesh  & \multirow{2}{*}{/ \{c, \paljstop, \paljfric, j\}\_} \\
         a $\to$ \aesh & \aesh{} $\to$ e & e $\to$ i & 
    \end{tabular}
\end{center}

\subsection{Vowels become nasalized after glottals}

\begin{center}
    V $\to$ \~{V} / \{\glotstop,h,\voih\}\_
\end{center}

A non-phonemic tonal downstep is also applied to the relevant syllable. Ordinarily, this non-phonemic downstep happens alongside the syllable marked with primary stress.

e.g., \espq{kohorto} \bripa{kho\rhot\,t} $\to$ \bripa{kh\~{o}\rhot\,t}, \espq{viandhaketa\^{j}o} \bripa{vj\~{\aesh}.dhak\prstr t\ahoh\ezh} $\to$ \bripa{vj\~{\aesh}\downstep dh\~{a}k\prstr t\ahoh\ezh}

\subsection{{\sc Lowlands:} Stops followed by a glottal fricative of the same voicing become fricatives. Fricatives followed by a glottal fricative assimilate to that fricative in voicing.}

\begin{center}
    \begin{tabular}{rrrrrrrl}
        p v $\to$ f & 
        t t\tiebar s z $\to$ s & 
        t\tiebar\esh{} \ezh{} $\to$ \esh & 
        \rette{} \retezh{} $\to$ \retesh & 
        c \paljfric{} $\to$ ç & 
        k $\to$ x &
        \multirow{2}{*}{/ \_h} 
        \\[0.1cm]
        b $\to$ p & 
        d $\to$ t & 
        d\tiebar z $\to$ t\tiebar s & 
        \retde{} $\to$ \rette & 
        \paljstop{} $\to$ c & 
        g $\to$ k
        \\[0.1cm]\midrule
        b f $\to$ v & 
        d d\tiebar z s $\to$ z & 
        d\tiebar\ezh{} \esh{} $\to$ \ezh & 
        \retde{} \retesh{} $\to$ \retezh & 
        \paljstop $\to$ \paljfric & 
        g x $\to$ \egh &
        \multirow{2}{*}{/ \_\voih} 
        \\[0.1cm]
        p $\to$ b & 
        t $\to$ d & 
        t\tiebar s $\to$ d\tiebar z & 
        \rette{} $\to$ \retde & 
        c $\to$ \paljstop{} & 
        k $\to$ g
    \end{tabular}
\end{center}

e.g., \espq{kohorto} \bripa{kh\~{o}\rhot\,t} $\to$ \bripa{xh\~{o}t}, \espq{viandhaketa\^{j}o} \bripa{vj\~{\aesh}\downstep dh\~{a}k\prstr t\ahoh\ezh} $\to$ \bripa{vj\~{\aesh}\downstep th\~{a}k\prstr t\ahoh\ezh}

\subsection{{\sc Lowlands:} Glottal fricatives deleted after obstruents}

e.g., \espq{kohorto} \bripa{xh\~{o}\rhot\,t} $\to$ \bripa{x\~{o}t}, \espq{viandhaketa\^{j}o} \bripa{vj\~{\aesh}\downstep th\~{a}k\prstr t\ahoh\ezh} $\to$ \bripa{vj\~{\aesh}\downstep t\~{a}k\prstr t\ahoh\ezh}

\newpage

\subsection{{\sc Highlands:} Obstruent-glottal clusters become ejectives}

\begin{center}
    \begin{tabular}{rrrrr}
        
        t\glotstop{} d\glotstop{} $\to$ t\jekt &
        t\tiebar s\glotstop{} d\tiebar z\glotstop{} $\to$ t\tiebar s\jekt &
        t\tiebar\esh\glotstop{} d\tiebar\ezh\glotstop{} $\to$ t\tiebar\esh\jekt &
        s\glotstop{} z\glotstop{} $\to$ s\jekt &
        \esh\glotstop{} \ezh\glotstop{} $\to$ \esh\jekt
        \\[0.1cm]
        \rette\glotstop{} \retde\glotstop{} $\to$ \rette\jekt &
        t\tiebar\retesh\glotstop{} d\tiebar\retezh\glotstop{} $\to$ t\tiebar\retesh\jekt &
        \retesh\glotstop{} \retezh\glotstop{} $\to$ \retesh\jekt &
        c\glotstop{} \paljstop\glotstop{} $\to$ c\jekt &
        ç\glotstop{} \paljfric\glotstop{} $\to$ ç\jekt
        \\[0.1cm]
        p\glotstop{} b\glotstop{} $\to$ p\jekt &
        f\glotstop{} v\glotstop{} $\to$ f\jekt &
        k\glotstop{} g\glotstop{} $\to$ k\jekt &
        x\glotstop{} \egh\glotstop{} $\to$ x\jekt
    \end{tabular}
\end{center}
e.g., \espq{kohorto} \bripa{kh\~{o}\rhot\,t} $\to$ \bripa{k\jekt\~{o}\rhot\,t}

\subsection{Glottal fricatives disappear when they precede consonants and become velar elsewhere}

\begin{center}
    \begin{tabular}{l}
        h \voih{} $\to$ \nm{} / \_C\\
        h \voih{} $\to$ x \egh
    \end{tabular}
\end{center}

\subsection{Velars fricatives become palatal when they follow a front vowel}

\begin{center}
    x \egh{} $\to$ ç \paljfric{} / \{i,e,y,ø,\oesh,\aesh\}\_
\end{center}

e.g., \espq{rehejmiĝi} \bripa{re\voih\~{e}\lgth\prstr mid\tiebar\ezh} $\to$ \bripa{re\egh\~{e}\lgth\prstr mid\tiebar\ezh}

\subsection{Nasal clusters assimilate to the final nasal in the cluster}

e.g., \espq{enmari\^{g}o} \bripa{enmwyd\tiebar\ezh} $\to$ \bripa{emmwyd\tiebar\ezh}

\subsection{Obstruent + \bripa{l} clusters change into single obstruents}

\begin{center}
    \begin{tabular}{lr}
        pl\vless{} bl $\to$ \bilaf{} \bilav &
        fl\vless{} sl\vless{} \esh l\vless{} $\to$ \latfric\\
        tl\vless{} dl $\to$ t\tiebar\latfric{} d\tiebar\latfrivoic &
        vl zl \ezh l $\to$ \latfrivoic\\
    \end{tabular}

    \begin{tabular}{rll}
        \multirow{2}{*}{kl\vless{} gl} & $\to$ k\tiebar Ɬ g\tiebar Ɬ\voicd & {\footnotesize\sc (Lowlands)}\\
        & $\to$ x \egh & {\footnotesize\sc (Highlands)}
    \end{tabular}
\end{center}

\subsection{Vowels become round after \bripa{w}}

\begin{center}
    \begin{tabular}{llll}
        i $\to$ y &
        e $\to$ ø &
        a $\to$ \ahoh &
        \aesh $\to$ \oesh
    \end{tabular}
\end{center}

\subsection{Consonant + \bripa{w} clusters become single consonants}

\begin{center}
    \begin{tabular}{rrl}
        pw bw mw $\to$ w &
        fw sw $\to$ \bilaf &
        vw zw $\to$ \bilav \\[0.1cm]
        \engma w \retna w \egna w $\to$ \engma &
        nw $\to$ n &
        kw gw $\to$ x \egh \\[0.1cm]
        \multicolumn{3}{c}{tw t\tiebar sw t\tiebar\esh w t\tiebar\retesh w cw $\to$ k} \\[0.1cm]
        \multicolumn{3}{c}{dw d\tiebar zw d\tiebar\ezh w d\tiebar\retezh w \paljstop w $\to$ g}
    \end{tabular}
\end{center}

\subsection{Obstruent + r clusters change into single obstruents}

\begin{center}
    \begin{tabular}{ll}
        pr\vless{} fr\vless{} $\to$ \bilaf{}$^{R}$ &
        br vr $\to$ \bilav{}$^{R}$\\[0.1cm]
        tr\vless{} dr $\to$ t\tiebar\retesh{} d\tiebar\retezh &
        kr\vless{} gr $\to$ kw gw \\[0.1cm]
        sr\vless{} \esh r\vless{} $\to$ \retesh &
        zr \ezh r $\to$ \retezh
    \end{tabular}
\end{center}

$^{R}$ indicates rhotacization of the next vowel.

\subsection{\bripa{l} and \bripa{r} merge to \bripa{\alvlap}}

\subsection{Vowels become rhotic after retroflex consonants}

\begin{center}
    V $\to$ V\rhot{} / \{t\tiebar\retesh,d\tiebar\retezh,\retesh,\retezh,\rette,\retde\}\_
\end{center}

\subsection{Retroflexes merge with palatoalveolars (or pure alveolars if no palatoalveolar is available)}

\begin{center}
    t\tiebar\retesh{} d\tiebar\retezh{} \retesh{} \retezh{} \rette{} \retde{} \retna{} $\to$ t\tiebar\esh{} d\tiebar\ezh{} \esh{} \ezh{} t d n
\end{center}

\subsection{{\sc Lowlands:} Palatal stops merge with palatoalveolars}

\begin{center}
    c \paljstop{} $\to$ t\tiebar\esh{} d\tiebar\ezh
\end{center}

\subsection{Consont + \bripa{j} clusters become single consonants}

\begin{center}
    \begin{tabular}{lll}
        pj fj $\to$ ç &
        bj vj $\to$ \paljfric &
        kj gj $\to$ c \paljstop \\[0.1cm]
        tj t\tiebar\esh j $\to$ t\tiebar\esh &
        dj d\tiebar\ezh j $\to$ d\tiebar\ezh &
        sj \esh j $\to$ \esh \\[0.1cm]
        zj \ezh j $\to$ \ezh &
        mj nj \egna j \engma j $\to$ \egna &
        \alvlap j $\to$ \latfrivoic
    \end{tabular}
\end{center}

\subsection{Stop plus nasal clusters that don't cross syllable boundaries become single nasals in the place-of-articulation of the original stop}

e.g., \espq{knabo} \bripa{knab} $\to$ \bripa{\egna ab}

\subsection{Palatals become velar when not adjacent to a front vowel}

\subsection{Velars become palatal before or after front vowels}

(in Lowlands, this surfaces as palatoalveolar affricates to correspond to the stops)

\subsection{Vowels shift to a saner system}

\begin{center}
    o \ahoh{} $\to$ \opno \\
    \aesh{} $\to$ e \\
    \oesh{} a $\to$ a \\
\end{center}

\subsection{Labiovelar fricatives merge into labials}

\begin{center}
    f v $\to$ \bilaf{} \bilav{}
\end{center}

\subsection{VOwels are backed and rounded before \bripa{w}}

\begin{center}
    \begin{tabular}{rl}
    i y $\to$ u &
    \multirow{2}{*}{/ \_w}\\[0.1cm]
    e ø a $\to$ \opno{}
    \end{tabular}
\end{center}

\end{document}