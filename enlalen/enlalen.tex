\documentclass[a4paper,11pt,oneside,openany]{memoir}
\usepackage{fontspec}
\usepackage{amsmath} % a pretty standard package to enlarge your inventory of symbols
\usepackage{amssymb} % another common package for symbols
\usepackage[hidelinks]{hyperref} % enables hyperlinks in your document (no worries -- they show up only on the screen. When you print a hard copy, the colored boxes aren't there)
\usepackage{url} % helps typeset URLs properly, typically with the command \url
\usepackage[margin=.75in]{geometry} % page layout
\usepackage{tikz}
\usepackage{tikz-qtree}
\usepackage{wrapfig}
\usepackage{subcaption}
\usepackage{booktabs} % creates beautiful and professional tables
\usepackage{multirow}
\usepackage{textcomp}
\usepackage{expex}
\usepackage{tablefootnote}
\usepackage[calc,english]{datetime2}
\usepackage{bookmark}
\usepackage{blindtext}
\usepackage{phonrule}

\setmainfont{Charis SIL}[CharacterVariant=43:1]
\newfontfamily\libert{Linux Libertine}
\DeclareTextFontCommand{\textlibert}{\libert}

%\setlength{\parindent}{2em}
\setlength{\parskip}{1ex}
%\linespread{1.1}
%---generic symbols---%
\newcommand{\dao}{$\to$}
\newcommand{\nm}{\symbol{"2205}}
\newcommand{\ra}{\textgreater}
\newcommand{\ipkt}{·}
\newcommand{\til}{\textasciitilde}
\newcommand{\langbr}{⟨}
\newcommand{\rangbr}{⟩}

\newcommand{\ortho}[1]{$\langle$#1$\rangle$}
\newcommand{\bripa}[1]{[#1]}
\newcommand{\phipa}[1]{/#1/}
\newcommand{\eng}[1]{`#1'}
\newcommand{\enl}[1]{\textlibert{#1}}
\newcommand{\enlq}[1]{«~\enl{#1}~»}
\newcommand{\proto}[1]{\textit{*#1}}

%---Lang Romanization---%
\newcommand{\ph}{φ}
\newcommand{\te}{ϑ}
\newcommand{\kh}{χ}
\newcommand{\pp}{π}
\newcommand{\ta}{τ}
\newcommand{\kk}{κ}

%---IPA---%
%consonants
\newcommand{\bilaf}{ɸ}
\newcommand{\bilav}{β}
\newcommand{\tht}{θ}
\newcommand{\labrox}{ʋ}
\newcommand{\latfric}{ɬ}
\newcommand{\latfrivoic}{ɮ}
\newcommand{\alvr}{ɹ}
\newcommand{\alvrap}{ɾ}
\newcommand{\alvlap}{ɺ}
\newcommand{\esh}{ʃ}
\newcommand{\ezh}{ʒ}
\newcommand{\alvpalesh}{ɕ}
\newcommand{\alvpalezh}{ʑ}
\newcommand{\paljstop}{ɟ}
\newcommand{\paljfric}{ʝ}
\newcommand{\egna}{ɲ}
\newcommand{\retesh}{ʂ}
\newcommand{\retezh}{ʐ}
\newcommand{\retna}{ɳ}
\newcommand{\egh}{ɣ}
\newcommand{\engma}{ŋ}
\newcommand{\vell}{ʟ}
\newcommand{\velr}{ʁ}
\newcommand{\velprox}{ɰ}
\newcommand{\uvux}{χ}
\newcommand{\pharox}{ʕ}
\newcommand{\glotstop}{ʔ}
%vowels
\newcommand{\frno}{ø}
\newcommand{\bari}{ɨ}
\newcommand{\unru}{ɯ}
\newcommand{\unro}{ɤ}
\newcommand{\eps}{ɛ}
\newcommand{\oeps}{œ}
\newcommand{\sche}{ɘ}
\newcommand{\schwa}{ə}
\newcommand{\centruh}{ɜ}
\newcommand{\opno}{ɔ}
\newcommand{\aesh}{æ}
\newcommand{\oesh}{ɶ}
\newcommand{\centra}{ɐ}
\newcommand{\ahoh}{ɒ}
%diacritics and modifiers
\newcommand{\asp}{ʰ}
\newcommand{\lab}{ʷ}
\newcommand{\pal}{ʲ}
\newcommand{\jekt}{ʼ}
\newcommand{\nav}{̃}
\newcommand{\rhot}{˞}
\newcommand{\sylb}{̩}
\newcommand{\vless}{̥}
\newcommand{\upvless}{̊}
\newcommand{\bck}{̠}
\newcommand{\lamino}{̻}
\newcommand{\apico}{̺}
\newcommand{\lowap}{̞}
\newcommand{\prstr}{ˈ}
\newcommand{\scstr}{ˌ}
\newcommand{\tiebar}{͡}
\newcommand{\lgth}{ː}
\newcommand{\linglab}{̼}

\lingset{glstyle=nlevel,numoffset=3em,textoffset=1.5em,exskip=.75ex,belowglpreambleskip=.25ex,aboveglftskip=.25ex}

\DTMnewdatestyle{eurodate}{%
    \renewcommand{\DTMdisplaydate}[4]{%
        \number##3.\nobreakspace%           day
        \DTMmonthname{##2}\nobreakspace%    month
        \number##1%                         year
    }%
    \renewcommand{\DTMDisplaydate}{\DTMdisplaydate}%
}

\DTMsetdatestyle{eurodate}

\newlength{\drop}% for my convenience
\newcommand*{\titleP}{\begingroup%
%\FSfont{5bo} % FontSite Bergamo (Bembo)
\drop = 0.12\textheight
\vspace*{\drop}
\begin{center}
{\huge A Grammar of}\\[\baselineskip]
{\HUGE\sc \lang}\par
\end{center}
\vspace*{3\drop}
{\large By {\sc Bethany E. Toma}}
\vfill
{\today}
\vspace*{0.5\drop}
\endgroup}

\maxsecnumdepth{subsection}

\newcommand{\lang}{Enłalen}
\newcommand{\parentlang}{Old Elvish}

\begin{document}

\begin{titlingpage}
\titleP
%\clearpage
\end{titlingpage}
\frontmatter

\chapter{Background \& Motivation}
\clearpage
\tableofcontents

\chapter{List of Glossing Abbreviations}

\mainmatter

\chapter{World \& Culture}

The term `\lang{}' reportedly comes from the \parentlang{} word \proto{hiːnolamaːlaʔiːna}, a word which roughly means `a long time ago (we) talked like this'. This perception of the language, as some ancient thing that has endured through countless generations unchanged, is common among speakers of \lang{}. However, the language itself and the culture of its speakers has varied dramatically throughout the years, more than its speakers tend to be willing to openly acknowledge.

\chapter{Phonology}

\section{Modern \lang{} phonology}

\subsection{Phoneme inventory}

\begin{table}[ht]
    \centering
    \begin{tabular}{@{}rcccc@{}}
                    & Labial & Alveolar & Palatal &    Back    \\ \cmidrule(l){2-5}
     Lenis Stop     &   p    &    t     &         &   k\bck    \\
     Aspirated Stop & p\asp  &  t\asp   &         & k\bck\asp  \\
     Ejective Stop  & p\jekt &  t\jekt  &         & k\bck\jekt \\
     Fricative      &   f    &    s     &         &   \uvux    \\
     Nasal          &   m    &    n     &         &            \\
     Lateral        &        &    l     &         &   ʟ\bck    \\
     Glide          &   w    &          &    j    &     ʕ      \\
     Vowel          &   o    &          &    i    &     a      
    \end{tabular}
    \caption{\lang{} Phoneme Inventory}
    \label{tab:phon-inv}
\end{table}

\subsection{Allophony \& phonotactics}

\begin{itemize}
    \item Vowels \phipa{i, o, a} become nasal \bripa{e\nav, o\nav, a\nav} when they precede a nasal
    \item Front vowels and semivowels \bripa{i, e\nav, j} become central \bripa{\bari, \sche\nav, \velprox} before dorsal sonorants \bripa{\vell\bck, w, \pharox, a} and after dorsal obstruents \bripa{k\bck, k\bck\asp, k\bck\jekt, \uvux}
    \item Back vowel \bripa{o} becomes central \bripa{\sche} before \bripa{j, i}
    \item Mid, non-front vowels \bripa{\sche, \sche\nav, o, o\nav} become high \bripa{\bari, \bari\nav, u, u\nav} after a high semivowel \bripa{j, w, \velprox}
    \item Mid vowels \bripa{o, o\nav, e\nav, \sche, \sche\nav} become low lax \bripa{\ahoh, \ahoh\nav, æ\nav, a, a\nav}
    \item Aspirated stops \phipa{p\asp, t\asp, k\bck\asp} become unaspirated \bripa{p, t, k\bck} before a nasal vowel.
    \item Nasal and lateral consonants \phipa{m, n, l, \vell\bck} assimilate in place-of-articulation to a following obstruent -- before labials, \phipa{l} becomes \bripa{l\linglab} and \phipa{\vell\bck} remains \bripa{\vell\bck}.
    \item \phipa{lw, \vell w} become \bripa{\alvr\lab, \velr\lab}.
    \item Fricatives \phipa{f, s, \uvux} are voiced \bripa{v, z, \velr} intervocalically and when following a voiced consonant.
    \item Unaspirated stops \phipa{p, t, k\bck} are voiced \bripa{b, d, g} intervocalically.
    \item Aspirated stops \phipa{p\asp, t\asp, k\bck\asp} become voiceless fricatives \bripa{f, \tht\bck, \uvux} intervocalically 
    \item Ejectives \phipa{p\jekt, t\jekt, k\bck\jekt} become unaspirated voiceless stops \bripa{p, t, k\bck} intervocalically 
\end{itemize}

\subsection{Romanization}

The `romanization' of \lang{} is a near-phonemic representation, with the exception of using \ortho{e} to represent \phipa{i} when it occurs before a nasal (where it is allophonically lowered).

\begin{table}[ht]
    \centering
    \begin{tabular}{@{}ccccccccccccccccccccccc@{}}
        p & t & k\bck & p\asp & t\asp & k\bck\asp & p\jekt & t\jekt & k\bck\jekt & f & s & \uvux & m & n & l & \vell\bck & w & j & \pharox & o & i & a & \bripa{e\nav} \\
        \enl{p} & \enl{t} & \enl{k} & \enl{\ph} & \enl{\te} & \enl{\kh} & \enl{\pp} & \enl{\ta} & \enl{\kk} & \enl{f} & \enl{s} & \enl{h} & \enl{m} & \enl{n} & \enl{l} & \enl{ł} & \enl{w} & \enl{y} & \enl{q} & \enl{o} & \enl{i} & \enl{a} & \enl{e}
    \end{tabular}
\end{table}

\section{\parentlang{} \& the \lang{} orthography}

\subsection{Phonology of \parentlang{}}

\begin{table}[ht]
    \begin{minipage}[b]{.55\linewidth}
    \centering
    \begin{tabular}{@{}rcccc@{}}
                  & Labial & Alveolar & Dorsal & Glottal  \\
        Stops     &   p    &    t     &   k    & \glotstop \\
        Non-Stops &   m    &   n  l   &        &    h      
    \end{tabular}
    \subcaption{Consonants}\label{tab:oldlang_cons}
    \end{minipage}
    \begin{minipage}[b]{.4\linewidth}
    \centering
    \begin{tabular}{@{}rcc@{}}
             &  Front   &   Back   \\
        High & i i\lgth & o o\lgth \\
        Low  & \multicolumn{2}{c}{a a\lgth}
    \end{tabular}
    \subcaption{Vowels}\label{tab:oldlang_vow}
    \end{minipage}
    \caption{\parentlang{} Phoneme Inventory}
    \label{tab:oldlang_phon}
\end{table}

\subsection{Sound changes and their effects on the orthography}

\begin{itemize}
    \item 
\end{itemize}

\subsection{Orthographic reform}

\chapter{Morphosyntax}

\section{`Parts of Speech'}

\section{Nominalization}

\section{Word order \& sentence structure}

\chapter{Semantics}

\section{Definiteness \& specificity in nominalizers}

%\part{Dictionary}

%\part{Texts \& Translations}

%\backmatter

\end{document}