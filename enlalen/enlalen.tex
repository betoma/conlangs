\documentclass[a4paper,11pt,oneside,openany]{memoir}

\usepackage{fontspec}
\usepackage{amsmath} % a pretty standard package to enlarge your inventory of symbols
\usepackage{amssymb} % another common package for symbols
\usepackage[hidelinks]{hyperref} % enables hyperlinks in your document (no worries -- they show up only on the screen. When you print a hard copy, the colored boxes aren't there)
\usepackage{url} % helps typeset URLs properly, typically with the command \url
\usepackage[margin=.75in]{geometry} % page layout
\usepackage{tikz}
\usepackage{tikz-qtree}
\usepackage{wrapfig}
%\usepackage{subcaption}
\usepackage{booktabs} % creates beautiful and professional tables
\usepackage{multicol}
\usepackage{multirow}
\usepackage{textcomp}
\usepackage{expex}
\usepackage{enumitem}
\usepackage{tablefootnote}
\usepackage[calc,english]{datetime2}
\usepackage{suffix}
\usepackage{afterpage}
\usepackage{bookmark}
\usepackage{blindtext}
\usepackage{phonrule}
\usepackage[glosses,indexonlyfirst,nonumberlist,toc,nomain,mcolblock,,nogroupskip]{leipzig}
\usepackage{etoolbox}

\setmainfont{Brill}

%---generic symbols---%
\newcommand{\dao}{$\to$}
\newcommand{\nm}{\symbol{"2205}}
\newcommand{\ra}{\textgreater}
\newcommand{\ipkt}{·}
\newcommand{\til}{\textasciitilde}
\newcommand{\langbr}{⟨}
\newcommand{\rangbr}{⟩}

\newcommand{\ortho}[1]{$\langle$#1$\rangle$}
\newcommand{\bripa}[1]{[#1]}
\newcommand{\phipa}[1]{/#1/}
\newcommand{\eng}[1]{`#1'}

%---IPA---%
%consonants
\newcommand{\bilaf}{ɸ}
\newcommand{\bilav}{β}
\newcommand{\tht}{θ}
\newcommand{\labrox}{ʋ}
\newcommand{\latfric}{ɬ}
\newcommand{\latfrivoic}{ɮ}
\newcommand{\darkl}{ɫ}
\newcommand{\alvr}{ɹ}
\newcommand{\alvrap}{ɾ}
\newcommand{\alvlap}{ɺ}
\newcommand{\esh}{ʃ}
\newcommand{\ezh}{ʒ}
\newcommand{\alvpalesh}{ɕ}
\newcommand{\alvpalezh}{ʑ}
\newcommand{\paljstop}{ɟ}
\newcommand{\paljfric}{ʝ}
\newcommand{\egna}{ɲ}
\newcommand{\retesh}{ʂ}
\newcommand{\retezh}{ʐ}
\newcommand{\retna}{ɳ}
\newcommand{\egh}{ɣ}
\newcommand{\engma}{ŋ}
\newcommand{\vell}{ʟ}
\newcommand{\velr}{ʁ}
\newcommand{\velprox}{ɰ}
\newcommand{\uvux}{χ}
\newcommand{\pharox}{ʕ}
\newcommand{\glotstop}{ʔ}
%vowels
\newcommand{\frno}{ø}
\newcommand{\bari}{ɨ}
\newcommand{\unru}{ɯ}
\newcommand{\unro}{ɤ}
\newcommand{\eps}{ɛ}
\newcommand{\oeps}{œ}
\newcommand{\sche}{ɘ}
\newcommand{\schwa}{ə}
\newcommand{\centruh}{ɜ}
\newcommand{\opno}{ɔ}
\newcommand{\aesh}{æ}
\newcommand{\oesh}{ɶ}
\newcommand{\centra}{ɐ}
\newcommand{\ahoh}{ɒ}
%diacritics and modifiers
\newcommand{\asp}{ʰ}
\newcommand{\lab}{ʷ}
\newcommand{\pal}{ʲ}
\newcommand{\jekt}{ʼ}
\newcommand{\nav}{̃}
\newcommand{\rhot}{˞}
\newcommand{\sylb}{̩}
\newcommand{\vless}{̥}
\newcommand{\upvless}{̊}
\newcommand{\bck}{̠}
\newcommand{\dwnwrd}{̞}
\newcommand{\upwrd}{̝}
\newcommand{\lamino}{̻}
\newcommand{\apico}{̺}
\newcommand{\lowap}{̞}
\newcommand{\prstr}{ˈ}
\newcommand{\scstr}{ˌ}
\newcommand{\tiebar}{͡}
\newcommand{\lgth}{ː}
\newcommand{\linglab}{̼}
%tone letters
\newcommand{\toneH}{˥}
\newcommand{\toneM}{˧}
\newcommand{\toneL}{˩}
\newcommand{\toneMH}{˦}
\newcommand{\toneML}{˨}

\definelingstyle{default}{glstyle=nlevel,numoffset=3em,textoffset=1.5em,exskip=.75ex,belowglpreambleskip=.25ex,aboveglftskip=.25ex,everyglft=\it}

\lingset{lingstyle=default}

\DTMnewdatestyle{eurodate}{%
    \renewcommand{\DTMdisplaydate}[4]{%
        \number##3.\nobreakspace%           day
        \DTMmonthname{##2}\nobreakspace%    month
        \number##1%                         year
    }%
    \renewcommand{\DTMDisplaydate}{\DTMdisplaydate}%
}

\DTMsetdatestyle{eurodate}

%-----DICT COMMANDS------
%------------------------

\makeatletter
\@beginparpenalty=10000
\makeatother

\newcounter{dictwordcount}
\newcounter{definition}

\newenvironment{entrylist}{
    \begin{description}[leftmargin=*]
    }{
    \end{description}
    }

\makeatletter
\newenvironment{simplentry}[1]%
    {%
    \item[#1]\hfill
    \protected@edef\@currentlabelname{#1}%
    \setcounter{definition}{0}%
    \begin{description}[align=right,labelwidth=*,font=\normalfont]
    }{%
    \end{description}
    }%
\makeatother

\makeatletter
\newenvironment{dictentry}[3][]%
    {%
    \item[#2]\ifthenelse{\isempty{#1}}{\hfill}{\:\bripa{#1}\hfill}\\
    {\footnotesize #3}
    \protected@edef\@currentlabelname{#2}%
    \setcounter{definition}{0}%
    \refstepcounter{dictwordcount}%
    \begin{description}[align=right,labelwidth=*,font=\normalfont]
    }{%
    \end{description}
    }%
\makeatother

\newenvironment{entrysublist}{
    \vspace{2ex}
    \item[]\textit{Compounds \& Phrasal Forms}
    \begin{entrylist}
        \small
    }{
    \end{entrylist}
    }

\newcommand{\dictdef}[2][]{\refstepcounter{definition}%
    \item[\thedefinition.] \ifthenelse{\isempty{#1}}{
        #2
    }{
        \textit{#1}\: #2
    }%
    }%

\WithSuffix\newcommand\dictdef*[2][]{%
    \item[] \ifthenelse{\isempty{#1}}{
        #2
    }{
        \textit{#1} #2
    }%
    }

\newcounter{protwordcount}

\makeatletter
\newenvironment{protentry}[1]%
    {%
    \item[\proto{#1}]\hfill
    \protected@edef\@currentlabelname{#1}%
    \setcounter{definition}{0}%
    \refstepcounter{protwordcount}%
    \begin{description}[align=right,labelwidth=*,font=\normalfont]
    }{%
    \end{description}
    }%
\makeatother

%----------------------------
%---------Glossaries---------
%----------------------------

\makenoidxglossaries

\newleipzig{test}{tst}{It's a test!}

\glssetwidest{CABBA}

\makeatletter
\newglossarystyle{fixed-mcols}{%
    \setglossarystyle{alttree}%
    \renewenvironment{theglossary}%
    {%
        \begin{multicols}{3}%
            \def\@gls@prevlevel{-1}%
%           \mbox{}\par
        }%
        {\par\end{multicols}}%
}
\makeatother

%---------------------------
%---------------------------

\newcommand{\enl}[1]{\textit{#1}}
\newcommand{\enlq}[1]{«~#1»}
\newcommand{\proto}[1]{\textit{*#1}}

%---Lang Romanization---%
%\newcommand{\ph}{φ}
%\newcommand{\te}{ϑ}
%\newcommand{\kh}{χ}
\newcommand{\ppa}{π}
\newcommand{\tta}{τ}
\newcommand{\kka}{κ}
\newcommand{\engga}{ng}
\newcommand{\Ppa}{Π}
\newcommand{\Tta}{Τ'}
\newcommand{\Kka}{Κ'}
\newcommand{\Engga}{Ng}

\newcommand{\suph}{$^\textsc{h}$}
\newcommand{\supglot}{$^\textsc{\glotstop}$}
\newcommand{\supho}{$^{\textsc{h}_{o}}$}
\newcommand{\supgloto}{$^{\textsc{\glotstop}_{o}}$}
\newcommand{\supha}{$^{\textsc{h}_{a}}$}
\newcommand{\supglota}{$^{\textsc{\glotstop}_{a}}$}
\newcommand{\suphi}{$^{\textsc{h}_{i}}$}
\newcommand{\supgloti}{$^{\textsc{\glotstop}_{i}}$}

\newcommand{\parentlang}{Enłalen}
\newcommand{\childlangone}{Daughterlang}
\newcommand{\childlangtwo}{Sonlang}
\newcommand{\childlangthree}{Kidlang}

\newlength{\drop}% for my convenience
\newcommand*{\titleP}{\begingroup%
%\FSfont{5bo} % FontSite Bergamo (Bembo)
\drop = 0.12\textheight
\vspace*{\drop}
\begin{center}
{\huge A Grammar of}\\[\baselineskip]
{\HUGE\sc \parentlang}\par
\end{center}
\vspace*{3\drop}
{\large By {\sc Bethany E. Toma}}
\vfill
{\today}
\vspace*{0.5\drop}
\endgroup}

\maxsecnumdepth{subsection}

%\setlength{\parindent}{2em}
\setlength{\parskip}{1ex}
%\linespread{1.1}

\begin{document}

\begin{titlingpage}
\titleP
%\clearpage
\end{titlingpage}
\frontmatter

\chapter{Background \& Motivation}
\clearpage
\tableofcontents

\setglossarystyle{fixed-mcols}

\printnoidxglossary[type=\leipzigtype,title={Glossing Abbreviations}]

\mainmatter

\part{\parentlang{} Grammar}

\chapter{Context \& Culture}

Placeholder \\
\Fsg \\
\Spl

\chapter{Phonology}

\section{Phonemics \& Allophony}

\subsection{Phoneme Inventory}

\begin{table}[h]
    \centering
    \begin{tabular}{@{}rccc@{}}
    \toprule
     & Labial & Coronal & Velar \\ \midrule
    Ejective & pʼ & tʼ & kʼ \\
    Plosive & p & t & k \\
    Fricative & f & s & \\
    Nasal & m & n & ŋ \\
    Liquid &  & l & ʟ \\ \bottomrule
    \end{tabular}
    \caption{Consonant Inventory}
    \label{tab:enl-consonants}
\end{table}

\begin{table}[h]
    \centering
    \begin{tabular}{@{}rccc@{}}
    \toprule
    \multicolumn{1}{l}{} & Front & Central & Back \\ \midrule
    High & i j &  &  \\
    Mid &  &  & o w \\
    Low &  & a \pharox\dwnwrd &  \\ \bottomrule
    \end{tabular}
    \caption{Vowel \& Glide Inventory}
    \label{tab:enl-vowels}
\end{table}

\parentlang{} possesses the glides \bripa{j, w, \pharox\dwnwrd} in certain positions, but these are analyzed as consonantal allophones of the vowel phonemes.

\subsection{Allophony \& Phonotactics}

To be replaced with better versions of these and such later:

\begin{itemize}
    \item Vowels \phipa{i, o, a} become nasal \bripa{e\nav, o\nav, a\nav} when they precede a nasal
    \item \phipa{l} becomes \bripa{\vell} before a velar consonant and \bripa{\bilav\dwnwrd} before a labial consonant. \phipa{w} is considered velar for these purposes.
    \item \phipa{\vell} becomes \bripa{\darkl} before a coronal consonant and \bripa{w} before a labial consonant. \phipa{w} is considered velar for these purposes.
    \item Front vowels and semivowels \bripa{i, e\nav, j} become central \bripa{\bari, \sche\nav, j} before non-palatal dorsal sonorants (including vowels) \bripa{\engma, \vell, w, \pharox\dwnwrd, a} and after non-palatal dorsal obstruents \bripa{k, k\jekt}
    \item Back vowel \bripa{o} becomes central \bripa{\sche} before \bripa{j, i(\lgth)}
    \item Mid, non-front vowels \bripa{\sche, \sche\nav, o, o\nav} become high \bripa{\bari, \bari\nav, u, u\nav} after a high semivowel \bripa{j, w, \velprox}
    \item Mid vowels \bripa{o, o\nav, e\nav, \sche, \sche\nav} become low lax \bripa{\ahoh, \ahoh\nav, \aesh\nav, a, a\nav} before non-palatal dorsal sonorants (\emph{not} including vowels) \bripa{\engma, \vell, w, \pharox\dwnwrd} and after non-palatal dorsal obstruents \bripa{k, k\jekt}.
    \item Nasals assimilate in place-of-articulation to a following obstruent.
    \item In Northern lects only:
    \begin{itemize}
        \item Standalone fricatives become debuccalized to \bripa{h} intervocalically. Geminate fricatives become single fricatives.
        \item Standalone voiceless stops become voiceless fricatives intervocalically. Geminate voiceless stops become single voiceless stops.
        \item Standalone ejectives become unaspirated voiceless stops intervocalically. Geminate ejectives become single ejectives.
    \end{itemize}
    \item In Southern lects only:
    \begin{itemize}
        \item Standalone voiceless fricatives become voiced intervocalically. Geminate voiceless fricatives become single voiceless fricatives.
        \item Standalone voiceless stops become voiced stops intervocalically. Geminate voiceless stops become single voiceless stops.
        \item Geminate ejectives become single ejectives. Standalone ejectives remain the same intervocalically.
    \end{itemize}
\end{itemize}

\section{Structure \& Suprasegmental Features}

\subsection{Syllable Structure}

\subsection{Stress}

%Stress in \parentlang{} is weight-sensitive, relying on a principle that classifies syllables containing long vowels as `heavy' and those without long vowels as `light.' Note that this is only the case for the stem---peripherals are considered light even when the nucleus is a long vowel. The determination of stress location is unbounded, so stress may appear anywhere in a given word, with right-headed assignment for heavy syllables and left-headed assignment for light syllables. The result of this is that stress is placed on the rightmost heavy syllable if there is one, or on the first syllable if there are none.

\subsection{Tone}

\section{Prosody}

\section{Transcription}

\subsection{Romanization}

Enłalen's romanization, used only for documentation purposes, is generally an unremarkable representation of the IPA-phonology, with a few exceptions:
\begin{itemize}
    \item \phipa{j} is written as \ortho{y}
    \item \phipa{\pharox\dwnwrd} is written as \ortho{r}
    \item \phipa{\vell} is written as \ortho{ł}
    \item \phipa{i} is written as \ortho{i} in most positions, but is written with \ortho{e} when it precedes a nasal and is realized as \bripa{ẽ}. Nasalization is not otherwise illustrated in the orthography.
    \item \phipa{\engma} is written as \ortho{n} in contexts in which it does not contrast with \phipa{n} (i.e., when it precedes velar phones as in \enl{enłalen} \bripa{ẽŋ\vell alẽn}) but is written as \ortho{\engga} in positions in which it contrasts with \phipa{n} (as in \enl{ale\engga} \bripa{alẽ\engma}).
    \item The ejective series \phipa{p\jekt{} t\jekt{} k\jekt{}} is written with the letters \ortho{\ppa{} \tta{} \kka{}}
    \item Phonemic high tone on a syllable is indicated by placing the acute diacritic on the vowel of that syllable, and phonemic low tone on a syllable is indicated by placing an underdot diacritic on the vowel of that syllable.
\end{itemize}

\subsection{Orthography}

\part{\childlangone{} Grammar}

\chapter{Context \& Culture}

\chapter{Phonology}

Daughterlang ideas to save for later:

\begin{itemize}
    \item \phipa{lw, \vell w} become \bripa{\alvr\lab, \velr\lab}
\end{itemize}

\subsection{Romanization}

\chapter{Morphosyntax}

\chapter{Semantics}

\part{Dictionary}

\setsecnumdepth{part}
\settocdepth{part}

\chapter{Old Elvish}

\begin{multicols*}{2}

\begin{entrylist}
    \begin{protentry}{hi\lgth lino}\label{OE:hiilino}
        \dictdef*{
            speech, talking, discussion, chatting, communication\\
            {\footnotesize Descendants: 
            \textit{\nameref{enl:enLalen_HLM}}
            }
        }
    \end{protentry}
    \begin{protentry}{hi\lgth na\glotstop ala\lgth}
    \label{OE:hiina7alaa}
        \dictdef*{
            the World, the physical realm, the mortal plane, this world\\
            {\footnotesize Descendants: 
            \textit{\nameref{enl:enLalen_HLM}}
            }
        }
    \end{protentry}
\end{entrylist}

\end{multicols*}

\chapter{En\l alen}

\begin{multicols*}{2}

\section{E}

\begin{entrylist}
    \begin{dictentry}[ẽ\engma\toneH\vell a\toneL lẽn\toneM]{énłạlen}{
        troch., from Old Elv. \proto{hi\lgth na\glotstop ala\lgth hili\lgth no} `World-speech' from \proto{\nameref{OE:hiina7alaa}} `the World' + \proto{\nameref{OE:hiilino}} `speech'
        }
        \label{enl:enLalen_HLM}
        \dictdef*[n.]{this language, En\l alen}
    \end{dictentry}
\end{entrylist}

\section{I}

\begin{entrylist}
    \begin{dictentry}[il\toneH jũn\toneM]{ílyon}{
        dacty., from Old Elv. \proto{\nameref{OE:hiilino}} `speech'
    }
        \dictdef*[v.]{
            utterance, statement, sentence
        }
        \begin{entrysublist}
            \begin{simplentry}{ẹn ílyon}
                \dictdef*[v.]{
                    to be sapient, to be intelligent, to be mentally aware, to have one's faculties (lit., to have speech)
                }
            \end{simplentry}
        \end{entrysublist}
    \end{dictentry}
\end{entrylist}

\end{multicols*}

\part{Texts \& Translations}

%\backmatter

\end{document}