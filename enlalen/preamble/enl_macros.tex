\newcommand{\enl}[1]{\textit{#1}}
\newcommand{\enlq}[1]{«~#1»}
\newcommand{\proto}[1]{\textit{*#1}}

%---Lang Romanization---%
%\newcommand{\ph}{φ}
%\newcommand{\te}{ϑ}
%\newcommand{\kh}{χ}
\newcommand{\ppa}{π}
\newcommand{\tta}{τ}
\newcommand{\kka}{κ}
\newcommand{\engga}{ng}

\newcommand{\parentlang}{Enłalen}
\newcommand{\childlangone}{Daughterlang}
\newcommand{\childlangtwo}{Sonlang}
\newcommand{\childlangthree}{Kidlang}

\newlength{\drop}% for my convenience
\newcommand*{\titleP}{\begingroup%
%\FSfont{5bo} % FontSite Bergamo (Bembo)
\drop = 0.12\textheight
\vspace*{\drop}
\begin{center}
{\huge A Grammar of}\\[\baselineskip]
{\HUGE\sc \parentlang}\par
\end{center}
\vspace*{3\drop}
{\large By {\sc Bethany E. Toma}}
\vfill
{\today}
\vspace*{0.5\drop}
\endgroup}